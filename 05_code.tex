% Options for packages loaded elsewhere
\PassOptionsToPackage{unicode}{hyperref}
\PassOptionsToPackage{hyphens}{url}
\PassOptionsToPackage{dvipsnames,svgnames,x11names}{xcolor}
%
\documentclass[
  letterpaper,
  DIV=11,
  numbers=noendperiod]{scrartcl}

\usepackage{amsmath,amssymb}
\usepackage{lmodern}
\usepackage{iftex}
\ifPDFTeX
  \usepackage[T1]{fontenc}
  \usepackage[utf8]{inputenc}
  \usepackage{textcomp} % provide euro and other symbols
\else % if luatex or xetex
  \usepackage{unicode-math}
  \defaultfontfeatures{Scale=MatchLowercase}
  \defaultfontfeatures[\rmfamily]{Ligatures=TeX,Scale=1}
\fi
% Use upquote if available, for straight quotes in verbatim environments
\IfFileExists{upquote.sty}{\usepackage{upquote}}{}
\IfFileExists{microtype.sty}{% use microtype if available
  \usepackage[]{microtype}
  \UseMicrotypeSet[protrusion]{basicmath} % disable protrusion for tt fonts
}{}
\makeatletter
\@ifundefined{KOMAClassName}{% if non-KOMA class
  \IfFileExists{parskip.sty}{%
    \usepackage{parskip}
  }{% else
    \setlength{\parindent}{0pt}
    \setlength{\parskip}{6pt plus 2pt minus 1pt}}
}{% if KOMA class
  \KOMAoptions{parskip=half}}
\makeatother
\usepackage{xcolor}
\setlength{\emergencystretch}{3em} % prevent overfull lines
\setcounter{secnumdepth}{5}
% Make \paragraph and \subparagraph free-standing
\ifx\paragraph\undefined\else
  \let\oldparagraph\paragraph
  \renewcommand{\paragraph}[1]{\oldparagraph{#1}\mbox{}}
\fi
\ifx\subparagraph\undefined\else
  \let\oldsubparagraph\subparagraph
  \renewcommand{\subparagraph}[1]{\oldsubparagraph{#1}\mbox{}}
\fi

\usepackage{color}
\usepackage{fancyvrb}
\newcommand{\VerbBar}{|}
\newcommand{\VERB}{\Verb[commandchars=\\\{\}]}
\DefineVerbatimEnvironment{Highlighting}{Verbatim}{commandchars=\\\{\}}
% Add ',fontsize=\small' for more characters per line
\usepackage{framed}
\definecolor{shadecolor}{RGB}{241,243,245}
\newenvironment{Shaded}{\begin{snugshade}}{\end{snugshade}}
\newcommand{\AlertTok}[1]{\textcolor[rgb]{0.68,0.00,0.00}{#1}}
\newcommand{\AnnotationTok}[1]{\textcolor[rgb]{0.37,0.37,0.37}{#1}}
\newcommand{\AttributeTok}[1]{\textcolor[rgb]{0.40,0.45,0.13}{#1}}
\newcommand{\BaseNTok}[1]{\textcolor[rgb]{0.68,0.00,0.00}{#1}}
\newcommand{\BuiltInTok}[1]{\textcolor[rgb]{0.00,0.23,0.31}{#1}}
\newcommand{\CharTok}[1]{\textcolor[rgb]{0.13,0.47,0.30}{#1}}
\newcommand{\CommentTok}[1]{\textcolor[rgb]{0.37,0.37,0.37}{#1}}
\newcommand{\CommentVarTok}[1]{\textcolor[rgb]{0.37,0.37,0.37}{\textit{#1}}}
\newcommand{\ConstantTok}[1]{\textcolor[rgb]{0.56,0.35,0.01}{#1}}
\newcommand{\ControlFlowTok}[1]{\textcolor[rgb]{0.00,0.23,0.31}{#1}}
\newcommand{\DataTypeTok}[1]{\textcolor[rgb]{0.68,0.00,0.00}{#1}}
\newcommand{\DecValTok}[1]{\textcolor[rgb]{0.68,0.00,0.00}{#1}}
\newcommand{\DocumentationTok}[1]{\textcolor[rgb]{0.37,0.37,0.37}{\textit{#1}}}
\newcommand{\ErrorTok}[1]{\textcolor[rgb]{0.68,0.00,0.00}{#1}}
\newcommand{\ExtensionTok}[1]{\textcolor[rgb]{0.00,0.23,0.31}{#1}}
\newcommand{\FloatTok}[1]{\textcolor[rgb]{0.68,0.00,0.00}{#1}}
\newcommand{\FunctionTok}[1]{\textcolor[rgb]{0.28,0.35,0.67}{#1}}
\newcommand{\ImportTok}[1]{\textcolor[rgb]{0.00,0.46,0.62}{#1}}
\newcommand{\InformationTok}[1]{\textcolor[rgb]{0.37,0.37,0.37}{#1}}
\newcommand{\KeywordTok}[1]{\textcolor[rgb]{0.00,0.23,0.31}{#1}}
\newcommand{\NormalTok}[1]{\textcolor[rgb]{0.00,0.23,0.31}{#1}}
\newcommand{\OperatorTok}[1]{\textcolor[rgb]{0.37,0.37,0.37}{#1}}
\newcommand{\OtherTok}[1]{\textcolor[rgb]{0.00,0.23,0.31}{#1}}
\newcommand{\PreprocessorTok}[1]{\textcolor[rgb]{0.68,0.00,0.00}{#1}}
\newcommand{\RegionMarkerTok}[1]{\textcolor[rgb]{0.00,0.23,0.31}{#1}}
\newcommand{\SpecialCharTok}[1]{\textcolor[rgb]{0.37,0.37,0.37}{#1}}
\newcommand{\SpecialStringTok}[1]{\textcolor[rgb]{0.13,0.47,0.30}{#1}}
\newcommand{\StringTok}[1]{\textcolor[rgb]{0.13,0.47,0.30}{#1}}
\newcommand{\VariableTok}[1]{\textcolor[rgb]{0.07,0.07,0.07}{#1}}
\newcommand{\VerbatimStringTok}[1]{\textcolor[rgb]{0.13,0.47,0.30}{#1}}
\newcommand{\WarningTok}[1]{\textcolor[rgb]{0.37,0.37,0.37}{\textit{#1}}}

\providecommand{\tightlist}{%
  \setlength{\itemsep}{0pt}\setlength{\parskip}{0pt}}\usepackage{longtable,booktabs,array}
\usepackage{calc} % for calculating minipage widths
% Correct order of tables after \paragraph or \subparagraph
\usepackage{etoolbox}
\makeatletter
\patchcmd\longtable{\par}{\if@noskipsec\mbox{}\fi\par}{}{}
\makeatother
% Allow footnotes in longtable head/foot
\IfFileExists{footnotehyper.sty}{\usepackage{footnotehyper}}{\usepackage{footnote}}
\makesavenoteenv{longtable}
\usepackage{graphicx}
\makeatletter
\def\maxwidth{\ifdim\Gin@nat@width>\linewidth\linewidth\else\Gin@nat@width\fi}
\def\maxheight{\ifdim\Gin@nat@height>\textheight\textheight\else\Gin@nat@height\fi}
\makeatother
% Scale images if necessary, so that they will not overflow the page
% margins by default, and it is still possible to overwrite the defaults
% using explicit options in \includegraphics[width, height, ...]{}
\setkeys{Gin}{width=\maxwidth,height=\maxheight,keepaspectratio}
% Set default figure placement to htbp
\makeatletter
\def\fps@figure{htbp}
\makeatother

\KOMAoption{captions}{tableheading}
\makeatletter
\makeatother
\makeatletter
\makeatother
\makeatletter
\@ifpackageloaded{caption}{}{\usepackage{caption}}
\AtBeginDocument{%
\ifdefined\contentsname
  \renewcommand*\contentsname{Table of contents}
\else
  \newcommand\contentsname{Table of contents}
\fi
\ifdefined\listfigurename
  \renewcommand*\listfigurename{List of Figures}
\else
  \newcommand\listfigurename{List of Figures}
\fi
\ifdefined\listtablename
  \renewcommand*\listtablename{List of Tables}
\else
  \newcommand\listtablename{List of Tables}
\fi
\ifdefined\figurename
  \renewcommand*\figurename{Figure}
\else
  \newcommand\figurename{Figure}
\fi
\ifdefined\tablename
  \renewcommand*\tablename{Table}
\else
  \newcommand\tablename{Table}
\fi
}
\@ifpackageloaded{float}{}{\usepackage{float}}
\floatstyle{ruled}
\@ifundefined{c@chapter}{\newfloat{codelisting}{h}{lop}}{\newfloat{codelisting}{h}{lop}[chapter]}
\floatname{codelisting}{Listing}
\newcommand*\listoflistings{\listof{codelisting}{List of Listings}}
\makeatother
\makeatletter
\@ifpackageloaded{caption}{}{\usepackage{caption}}
\@ifpackageloaded{subcaption}{}{\usepackage{subcaption}}
\makeatother
\makeatletter
\@ifpackageloaded{tcolorbox}{}{\usepackage[many]{tcolorbox}}
\makeatother
\makeatletter
\@ifundefined{shadecolor}{\definecolor{shadecolor}{rgb}{.97, .97, .97}}
\makeatother
\makeatletter
\makeatother
\ifLuaTeX
  \usepackage{selnolig}  % disable illegal ligatures
\fi
\IfFileExists{bookmark.sty}{\usepackage{bookmark}}{\usepackage{hyperref}}
\IfFileExists{xurl.sty}{\usepackage{xurl}}{} % add URL line breaks if available
\urlstyle{same} % disable monospaced font for URLs
\hypersetup{
  pdftitle={Model Code},
  pdfauthor={Severin Reissl and Luis Sarmiento},
  colorlinks=true,
  linkcolor={blue},
  filecolor={Maroon},
  citecolor={Blue},
  urlcolor={Blue},
  pdfcreator={LaTeX via pandoc}}

\title{Model Code}
\author{Severin Reissl and Luis Sarmiento}
\date{}

\begin{document}
\maketitle
\ifdefined\Shaded\renewenvironment{Shaded}{\begin{tcolorbox}[frame hidden, interior hidden, enhanced, boxrule=0pt, borderline west={3pt}{0pt}{shadecolor}, sharp corners, breakable]}{\end{tcolorbox}}\fi

\renewcommand*\contentsname{Table of contents}
{
\hypersetup{linkcolor=}
\setcounter{tocdepth}{3}
\tableofcontents
}
The model is written in Python. Below we provide a step-by-step
documentation of all model functions. In total the model consists of 10
separate functions, which are in turn called by one main model function
containing the initialisation and the main simulation loop. This main
function is described last.

\hypertarget{generate-electricity-price}{%
\section{Generate Electricity Price}\label{generate-electricity-price}}

This function is called once at the beginning of a simulation run in
order to generate a time-series of the electricity price which agents
have to pay at each time-step of the simulation (note that this is
different from the expected future electricity price, described below).

The function takes four inputs. \texttt{\_PriceEmp} is a pandas
dataframe containing an empirical time-series of the electricity price.
\texttt{\_Params}, is a pandas dataframe containing the names and values
of the model parameters. \texttt{\_length}, is an integer denoting the
length of the time-series to be created (given by simulation length+2
since we need two lagged values at the beginning for the expectations
formation process discussed below). Finally, \texttt{\_start} denotes
the start year of the simulation

The function then creates two numpy vectors of zeros of size
\texttt{\_length}. \texttt{electricityprice1} is the vector of prices
that will actually be used in the simulation. \texttt{electricityprice2}
instead is a vector of prices which will be used to train the agents'
expectations formation mechanism prior to the beginning of the
simulation.

The first two elements of the vectors are set to the empirical values of
the electricity price in \texttt{\_start-2} and \texttt{\_start-1}. The
rest of the vector \texttt{electricityprice1} is then filled either with
empirical values of the price (if these are available, e.g.~because we
are simulating past years rather than the future) or by iterating
forward an AR(1) model previously estimated on the empirical price data,
including a normally distributed random shock the standard deviation of
which is taken from the standard deviation of the residuals from the
estimation of the AR(1) model. \texttt{electricityprice2}, the
``training set'', is filled completely with simulated values from the
AR(1) model. Finally, the function returns the two time-series of the
electricity price.

\begin{Shaded}
\begin{Highlighting}[]
\KeywordTok{def}\NormalTok{ GenerateElectricityPrice(\_PriceEmp,\_Params,\_length,\_start):}
    \CommentTok{\#Create vectors of zeros}
\NormalTok{    electricityprice1}\OperatorTok{=}\NormalTok{np.zeros(\_length)}
\NormalTok{    electricityprice2}\OperatorTok{=}\NormalTok{np.zeros(\_length)}
    \CommentTok{\#Set first two elements to empirical electricity price in \_start{-}2 and \_start{-}1}
\NormalTok{    electricityprice1[}\DecValTok{0}\NormalTok{]}\OperatorTok{=}\NormalTok{\_PriceEmp[}\BuiltInTok{str}\NormalTok{(\_start}\OperatorTok{{-}}\DecValTok{2}\NormalTok{)].values[}\DecValTok{0}\NormalTok{]}
\NormalTok{    electricityprice2[}\DecValTok{0}\NormalTok{]}\OperatorTok{=}\NormalTok{\_PriceEmp[}\BuiltInTok{str}\NormalTok{(\_start}\OperatorTok{{-}}\DecValTok{2}\NormalTok{)].values[}\DecValTok{0}\NormalTok{]}
\NormalTok{    electricityprice1[}\DecValTok{1}\NormalTok{]}\OperatorTok{=}\NormalTok{\_PriceEmp[}\BuiltInTok{str}\NormalTok{(\_start}\OperatorTok{{-}}\DecValTok{1}\NormalTok{)].values[}\DecValTok{0}\NormalTok{]}
\NormalTok{    electricityprice2[}\DecValTok{1}\NormalTok{]}\OperatorTok{=}\NormalTok{\_PriceEmp[}\BuiltInTok{str}\NormalTok{(\_start}\OperatorTok{{-}}\DecValTok{1}\NormalTok{)].values[}\DecValTok{0}\NormalTok{]}
    \CommentTok{\#Iterate over rest of vector}
    \ControlFlowTok{for}\NormalTok{ t }\KeywordTok{in} \BuiltInTok{range}\NormalTok{(}\DecValTok{2}\NormalTok{,\_length):}
        \CommentTok{\#If the empirical price{-}series still contains values, use them to fill electricityprice1}
        \ControlFlowTok{if} \BuiltInTok{sum}\NormalTok{(}\BuiltInTok{str}\NormalTok{(\_start}\OperatorTok{{-}}\DecValTok{2}\OperatorTok{+}\NormalTok{t)}\OperatorTok{==}\NormalTok{\_PriceEmp.columns)}\OperatorTok{==}\DecValTok{1}\NormalTok{:}
\NormalTok{            electricityprice1[t]}\OperatorTok{=}\NormalTok{\_PriceEmp[}\BuiltInTok{str}\NormalTok{(\_start}\OperatorTok{{-}}\DecValTok{2}\OperatorTok{+}\NormalTok{t)].values[}\DecValTok{0}\NormalTok{]}
        \ControlFlowTok{else}\NormalTok{:}
        \CommentTok{\#Otherwise, fill the rest of electricityprice1 with simulated prices using a previously estimated AR1 model}
            \CommentTok{\#Generate normally distributed shock based on standard deviation of regression residuals}
\NormalTok{            shock1}\OperatorTok{=}\NormalTok{\_Params[}\StringTok{"ElectricityPriceSD"}\NormalTok{].values[}\DecValTok{0}\NormalTok{]}\OperatorTok{*}\NormalTok{np.sqrt(}\OperatorTok{{-}}\DecValTok{2}\OperatorTok{*}\NormalTok{np.log(np.random.uniform(}\DecValTok{0}\NormalTok{,}\DecValTok{1}\NormalTok{,}\DecValTok{1}\NormalTok{)))}\OperatorTok{*}\NormalTok{np.cos(}\DecValTok{2}\OperatorTok{*}\NormalTok{np.pi}\OperatorTok{*}\NormalTok{np.random.uniform(}\DecValTok{0}\NormalTok{,}\DecValTok{1}\NormalTok{,}\DecValTok{1}\NormalTok{))}
            \CommentTok{\#Price in t calculated by iterating AR(1) model}
\NormalTok{            electricityprice1[t]}\OperatorTok{=}\NormalTok{electricityprice1[t}\OperatorTok{{-}}\DecValTok{1}\NormalTok{]}\OperatorTok{+}\NormalTok{\_Params[}\StringTok{"ElectricityPriceTrend"}\NormalTok{].values[}\DecValTok{0}\NormalTok{]}\OperatorTok{+}\NormalTok{\_Params[}\StringTok{"ElectricityPriceAR"}\NormalTok{].values[}\DecValTok{0}\NormalTok{]}\OperatorTok{*}\NormalTok{(electricityprice1[t}\OperatorTok{{-}}\DecValTok{1}\NormalTok{]}\OperatorTok{{-}}\NormalTok{electricityprice1[t}\OperatorTok{{-}}\DecValTok{2}\NormalTok{])}\OperatorTok{+}\NormalTok{shock1 }
        \CommentTok{\#electricityprice2, which is used to train the expectations formation mechanism, is filled completely with simulated values}
        \CommentTok{\#Generate normally distributed shock based on standard deviation of regression residuals}
\NormalTok{        shock2}\OperatorTok{=}\NormalTok{\_Params[}\StringTok{"ElectricityPriceSD"}\NormalTok{].values[}\DecValTok{0}\NormalTok{]}\OperatorTok{*}\NormalTok{np.sqrt(}\OperatorTok{{-}}\DecValTok{2}\OperatorTok{*}\NormalTok{np.log(np.random.uniform(}\DecValTok{0}\NormalTok{,}\DecValTok{1}\NormalTok{,}\DecValTok{1}\NormalTok{)))}\OperatorTok{*}\NormalTok{np.cos(}\DecValTok{2}\OperatorTok{*}\NormalTok{np.pi}\OperatorTok{*}\NormalTok{np.random.uniform(}\DecValTok{0}\NormalTok{,}\DecValTok{1}\NormalTok{,}\DecValTok{1}\NormalTok{))}
        \CommentTok{\#Price in t calculated by iterating AR(1) model}
\NormalTok{        electricityprice2[t]}\OperatorTok{=}\NormalTok{electricityprice2[t}\OperatorTok{{-}}\DecValTok{1}\NormalTok{]}\OperatorTok{+}\NormalTok{\_Params[}\StringTok{"ElectricityPriceTrend"}\NormalTok{].values[}\DecValTok{0}\NormalTok{]}\OperatorTok{+}\NormalTok{\_Params[}\StringTok{"ElectricityPriceAR"}\NormalTok{].values[}\DecValTok{0}\NormalTok{]}\OperatorTok{*}\NormalTok{(electricityprice2[t}\OperatorTok{{-}}\DecValTok{1}\NormalTok{]}\OperatorTok{{-}}\NormalTok{electricityprice2[t}\OperatorTok{{-}}\DecValTok{2}\NormalTok{])}\OperatorTok{+}\NormalTok{shock2}
    \ControlFlowTok{return}\NormalTok{ [electricityprice1,electricityprice2]}
\end{Highlighting}
\end{Shaded}

\hypertarget{set-policy}{%
\section{Set Policy}\label{set-policy}}

Depending on the values of various dummies in the \texttt{\_Params}
dataframe and the current simulation period \texttt{\_year}, this
function determines whether a policy experiment should be activated in
the current period.

At present, the model includes four possible policies, namely a subsidy
on the cost of a solar panel (\texttt{Subsidy1}), a subsidy on the price
at which electricity generated by PV can be sold to the grid
(\texttt{Subsidy2}), a policy increasing the maximum permitted loan to
value ratio for bank loans to acquire PV (\texttt{CreditPolicy1},
currently not used since in the baseline the maximum LTV is already set
to 1), and a policy increasing the maximum allowed debt service to
income ratio resulting from a loan made to acquire PV
(\texttt{CreditPolicy2}). For instance, if the dummy \texttt{\_Subsidy1}
is set to 1 and if the current simulation period (expressed as a year)
is equal to or larger than \texttt{SubsidyStart}, a subsidy on the
installation cost of PV will be activated. The function checks the
conditions for the respective policy variables in each period, changes
them if necessary, and then returns them.

\begin{Shaded}
\begin{Highlighting}[]
\KeywordTok{def}\NormalTok{ SetPolicy(\_Subsidy1,\_Subsidy2,\_CreditPolicy1,\_CreditPolicy2,\_year,\_Params):}
    \ControlFlowTok{if}\NormalTok{ \_Params[}\StringTok{"Subsidy1"}\NormalTok{].values[}\DecValTok{0}\NormalTok{]}\OperatorTok{==}\DecValTok{1} \OperatorTok{\&}\NormalTok{ \_year}\OperatorTok{\textgreater{}=}\NormalTok{\_Params[}\StringTok{"SubsidyStart"}\NormalTok{].values[}\DecValTok{0}\NormalTok{]:}
\NormalTok{        \_Subsidy1}\OperatorTok{=}\NormalTok{\_Params[}\StringTok{"s1"}\NormalTok{].values[}\DecValTok{0}\NormalTok{]}
    \ControlFlowTok{if}\NormalTok{ \_Params[}\StringTok{"Subsidy2"}\NormalTok{].values[}\DecValTok{0}\NormalTok{]}\OperatorTok{==}\DecValTok{1} \OperatorTok{\&}\NormalTok{ \_year}\OperatorTok{\textgreater{}=}\NormalTok{\_Params[}\StringTok{"SubsidyStart"}\NormalTok{].values[}\DecValTok{0}\NormalTok{]:}
\NormalTok{        \_Subsidy2}\OperatorTok{=}\NormalTok{\_Params[}\StringTok{"s2"}\NormalTok{].values[}\DecValTok{0}\NormalTok{]}
    \ControlFlowTok{if}\NormalTok{ \_Params[}\StringTok{"CreditPolicy1"}\NormalTok{].values[}\DecValTok{0}\NormalTok{]}\OperatorTok{==}\DecValTok{1} \OperatorTok{\&}\NormalTok{ \_year}\OperatorTok{\textgreater{}=}\NormalTok{\_Params[}\StringTok{"CreditPolicyStart"}\NormalTok{].values[}\DecValTok{0}\NormalTok{]:}
\NormalTok{        \_CreditPolicy1}\OperatorTok{=}\NormalTok{\_Params[}\StringTok{"cp1"}\NormalTok{].values[}\DecValTok{0}\NormalTok{]}
    \ControlFlowTok{if}\NormalTok{ \_Params[}\StringTok{"CreditPolicy2"}\NormalTok{].values[}\DecValTok{0}\NormalTok{]}\OperatorTok{==}\DecValTok{1} \OperatorTok{\&}\NormalTok{ \_year}\OperatorTok{\textgreater{}=}\NormalTok{\_Params[}\StringTok{"CreditPolicyStart"}\NormalTok{].values[}\DecValTok{0}\NormalTok{]:}
\NormalTok{        \_CreditPolicy2}\OperatorTok{=}\NormalTok{\_Params[}\StringTok{"cp2"}\NormalTok{].values[}\DecValTok{0}\NormalTok{]}
    \ControlFlowTok{return}\NormalTok{[\_Subsidy1,\_Subsidy2,\_CreditPolicy1,\_CreditPolicy2]}
\end{Highlighting}
\end{Shaded}

\hypertarget{income-and-power-cost}{%
\section{Income and Power cost}\label{income-and-power-cost}}

This function generates the incomes and electricity costs for each agent
in every period. \texttt{\_Income\_f} and \texttt{\_Powercost\_f} are
vectors of length \emph{N}, where \emph{N} is the number of agents in
the model. The deterministic values of income and power cost for each
agent are assumed to grow at rate \texttt{\_Trend} (which may be
heterogeneous across agents but is currently uniform) at each time step.
The electricity cost to be paid by each agent is deterministic, while
their income in the current period is drawn from a normal distribution
with mean \texttt{\_Income\_f} and standard deviation
\texttt{\_Income\_sd}.

Having determined income and power cost for each agent, we move on to
calculate the revenue/cost saving generated by solar panels in the
current period. We begin by setting \texttt{panelrevenue} as a copy of
\texttt{\_PV}, a vector of dummies indicating whether an agent owns PV
in the current period. Hence \texttt{panelrevenue} is zero (and will
remain zero in the calculations below) for agents who do not own PV.

We then calculate \texttt{panelrevenue} for the agents who do own PV, by
dividing them into 4 groups depending on whether their own electricity
consumption in the current period is larger or smaller than the capacity
of the solar panel and whether their feed-in tariff (FIT; possibly
including an additional subsidy if one is in force, see the function
above) is higher or lower than the current electricity price (note that,
as in reality in Germany, the feed-in tariff depends on the year in
which the PV was installed and remains constant thereafter, meaning that
agents will receive different FITs depending on how old their panel is).
\texttt{panelrevenue} is then set for each of the four groups, assuming
that agents can consume a maximum of
\texttt{\_Params{[}"PVCons"{]}.values{[}0{]}} of the amount of
electricity generated by the panel.

The electricity cost of agents owning PV is then reduced by
\texttt{panelrevenue} (note that this can lead to cases in which
electricity cost becomes smaller than 0 when the revenue from the panel
exceeds electricity cost). In addition, \texttt{\_CumulativeProfit},
which keeps track of the profit which PV owners have derived from their
panel over its lifetime, is updated. Finally, we calculate a measure of
income net of electricity cost which is used below to calculate income
distribution statistics.

\begin{Shaded}
\begin{Highlighting}[]
\KeywordTok{def}\NormalTok{ IncomePowercost(\_Income\_f,\_Trend,\_Powercost\_f,\_Income\_sd,\_CumulativeProfit,\_PV,\_Price,\_PanelFIT,\_Subsidy2,\_Params):}
    \CommentTok{\#Deterministic income and power cost values grow at exogenous trend}
\NormalTok{    \_Income\_f}\OperatorTok{=}\NormalTok{\_Income\_f}\OperatorTok{*}\NormalTok{(}\DecValTok{1}\OperatorTok{+}\NormalTok{\_Trend)}
\NormalTok{    \_Powercost\_f}\OperatorTok{=}\NormalTok{\_Powercost\_f}\OperatorTok{*}\NormalTok{(}\DecValTok{1}\OperatorTok{+}\NormalTok{\_Trend)}
    \CommentTok{\#Actual income is drawn from a normal distribution with mean \_Income\_f and standard deviation \_Income\_sd}
\NormalTok{    randincome}\OperatorTok{=}\NormalTok{np.random.uniform(}\DecValTok{0}\NormalTok{,}\DecValTok{1}\NormalTok{,\_Income\_f.size)}
\NormalTok{    randincome2}\OperatorTok{=}\NormalTok{np.random.uniform(}\DecValTok{0}\NormalTok{,}\DecValTok{1}\NormalTok{,\_Income\_f.size)}
\NormalTok{    randincome}\OperatorTok{=}\NormalTok{np.sqrt(}\OperatorTok{{-}}\DecValTok{2}\OperatorTok{*}\NormalTok{np.log(randincome))}\OperatorTok{*}\NormalTok{np.cos(}\DecValTok{2}\OperatorTok{*}\NormalTok{np.pi}\OperatorTok{*}\NormalTok{randincome2)}
\NormalTok{    randincome}\OperatorTok{=}\NormalTok{\_Income\_sd}\OperatorTok{*}\NormalTok{randincome}\OperatorTok{*}\NormalTok{\_Income\_f}\OperatorTok{+}\NormalTok{\_Income\_f}
\NormalTok{    randincome}\OperatorTok{=}\NormalTok{np.maximum(}\FloatTok{1e{-}10}\NormalTok{,randincome)}
    \CommentTok{\#Power cost is multiplied by current electricity price}
\NormalTok{    powercost}\OperatorTok{=}\NormalTok{\_Powercost\_f}\OperatorTok{*}\NormalTok{\_Price}
    \CommentTok{\#Calculate revenue/cost saving generated by existing solar panels in the current period. Note that if \_PV=0, panelrevenue=0}
\NormalTok{    panelrevenue}\OperatorTok{=}\NormalTok{\_PV.copy()}
    \CommentTok{\#Agents who own PV, whose electricity consumption exceeds the capacity of the panel and whose FIT is smaller than the current electricity price}
\NormalTok{    group1}\OperatorTok{=}\NormalTok{np.where((\_PV}\OperatorTok{==}\DecValTok{1}\NormalTok{) }\OperatorTok{\&}\NormalTok{ (\_Powercost\_f}\OperatorTok{\textgreater{}=}\NormalTok{\_Params[}\StringTok{"PVCapacity"}\NormalTok{].values[}\DecValTok{0}\NormalTok{]) }\OperatorTok{\&}\NormalTok{ ((\_PanelFIT}\OperatorTok{+}\NormalTok{\_Subsidy2)}\OperatorTok{\textless{}}\NormalTok{\_Price))}
    \CommentTok{\#Agents who own PV, whose electricity consumption exceeds the capacity of the panel and whose FIT is larger than the current electricity price}
\NormalTok{    group2}\OperatorTok{=}\NormalTok{np.where((\_PV}\OperatorTok{==}\DecValTok{1}\NormalTok{) }\OperatorTok{\&}\NormalTok{ (\_Powercost\_f}\OperatorTok{\textgreater{}=}\NormalTok{\_Params[}\StringTok{"PVCapacity"}\NormalTok{].values[}\DecValTok{0}\NormalTok{]) }\OperatorTok{\&}\NormalTok{ ((\_PanelFIT}\OperatorTok{+}\NormalTok{\_Subsidy2)}\OperatorTok{\textgreater{}=}\NormalTok{\_Price))}
    \CommentTok{\#Agents who own PV, whose electricity consumption is smaller than the capacity of the panel and whose FIT is smaller than the current electricity price}
\NormalTok{    group3}\OperatorTok{=}\NormalTok{np.where((\_PV}\OperatorTok{==}\DecValTok{1}\NormalTok{) }\OperatorTok{\&}\NormalTok{ (\_Powercost\_f}\OperatorTok{\textless{}}\NormalTok{\_Params[}\StringTok{"PVCapacity"}\NormalTok{].values[}\DecValTok{0}\NormalTok{]) }\OperatorTok{\&}\NormalTok{ ((\_PanelFIT}\OperatorTok{+}\NormalTok{\_Subsidy2)}\OperatorTok{\textless{}}\NormalTok{\_Price))}
    \CommentTok{\#Agents who own PV, whose electricity consumption is smaller than the capacity of the panel and whose FIT is larger than the current electricity price}
\NormalTok{    group4}\OperatorTok{=}\NormalTok{np.where((\_PV}\OperatorTok{==}\DecValTok{1}\NormalTok{) }\OperatorTok{\&}\NormalTok{ (\_Powercost\_f}\OperatorTok{\textless{}}\NormalTok{\_Params[}\StringTok{"PVCapacity"}\NormalTok{].values[}\DecValTok{0}\NormalTok{]) }\OperatorTok{\&}\NormalTok{ ((\_PanelFIT}\OperatorTok{+}\NormalTok{\_Subsidy2)}\OperatorTok{\textgreater{}=}\NormalTok{\_Price))}
    \CommentTok{\#Revenue of group 1 is given by share of own consumption times capacity times price plus electricity fed into the grid times FIT}
\NormalTok{    panelrevenue[group1]}\OperatorTok{=}\NormalTok{\_Params[}\StringTok{"PVCons"}\NormalTok{].values[}\DecValTok{0}\NormalTok{]}\OperatorTok{*}\NormalTok{\_Params[}\StringTok{"PVCapacity"}\NormalTok{].values[}\DecValTok{0}\NormalTok{]}\OperatorTok{*}\NormalTok{\_Price}\OperatorTok{+}\NormalTok{(}\DecValTok{1}\OperatorTok{{-}}\NormalTok{\_Params[}\StringTok{"PVCons"}\NormalTok{].values[}\DecValTok{0}\NormalTok{])}\OperatorTok{*}\NormalTok{\_Params[}\StringTok{"PVCapacity"}\NormalTok{].values[}\DecValTok{0}\NormalTok{]}\OperatorTok{*}\NormalTok{(\_PanelFIT[group1]}\OperatorTok{+}\NormalTok{\_Subsidy2)}
    \CommentTok{\#Revenue of group 2 is given by capacity times FIT}
\NormalTok{    panelrevenue[group2]}\OperatorTok{=}\NormalTok{\_Params[}\StringTok{"PVCapacity"}\NormalTok{].values[}\DecValTok{0}\NormalTok{]}\OperatorTok{*}\NormalTok{(\_PanelFIT[group2]}\OperatorTok{+}\NormalTok{\_Subsidy2)}
    \CommentTok{\#Revenue of group 3 is given by minimum between own consumption share times capacity times price and electricity consumption times price, plus the remaining capacity of the panel times FIT}
\NormalTok{    panelrevenue[group3]}\OperatorTok{=}\NormalTok{np.minimum(\_Params[}\StringTok{"PVCons"}\NormalTok{].values[}\DecValTok{0}\NormalTok{]}\OperatorTok{*}\NormalTok{\_Params[}\StringTok{"PVCapacity"}\NormalTok{].values[}\DecValTok{0}\NormalTok{],\_Powercost\_f[group3])}\OperatorTok{*}\NormalTok{\_Price}\OperatorTok{+}\NormalTok{(\_PanelFIT[group3]}\OperatorTok{+}\NormalTok{\_Subsidy2)}\OperatorTok{*}\NormalTok{(\_Params[}\StringTok{"PVCapacity"}\NormalTok{].values[}\DecValTok{0}\NormalTok{]}\OperatorTok{{-}}\NormalTok{np.minimum(\_Params[}\StringTok{"PVCons"}\NormalTok{].values[}\DecValTok{0}\NormalTok{]}\OperatorTok{*}\NormalTok{\_Params[}\StringTok{"PVCapacity"}\NormalTok{].values[}\DecValTok{0}\NormalTok{],\_Powercost\_f[group3]))}
    \CommentTok{\#Revenue of group 4 is given by capacity times FIT}
\NormalTok{    panelrevenue[group4]}\OperatorTok{=}\NormalTok{\_Params[}\StringTok{"PVCapacity"}\NormalTok{].values[}\DecValTok{0}\NormalTok{]}\OperatorTok{*}\NormalTok{(\_PanelFIT[group4]}\OperatorTok{+}\NormalTok{\_Subsidy2)}
    \CommentTok{\#Reduce electricity cost by the revenue/cost saving generated by solar panels}
\NormalTok{    powercost}\OperatorTok{=}\NormalTok{powercost}\OperatorTok{{-}}\NormalTok{panelrevenue}
    \CommentTok{\#Update cumulative profit of PV owners}
\NormalTok{    \_CumulativeProfit[\_PV}\OperatorTok{==}\DecValTok{1}\NormalTok{]}\OperatorTok{=}\NormalTok{\_CumulativeProfit[\_PV}\OperatorTok{==}\DecValTok{1}\NormalTok{]}\OperatorTok{+}\NormalTok{panelrevenue[\_PV}\OperatorTok{==}\DecValTok{1}\NormalTok{]}
    \CommentTok{\#Calculate income net of power cost}
\NormalTok{    income\_p}\OperatorTok{=}\NormalTok{randincome}\OperatorTok{{-}}\NormalTok{powercost}
\NormalTok{    income\_p}\OperatorTok{=}\NormalTok{np.maximum(}\DecValTok{0}\NormalTok{,income\_p)}
    \ControlFlowTok{return}\NormalTok{[randincome,powercost,income\_p,\_CumulativeProfit,\_Income\_f,\_Powercost\_f]}
\end{Highlighting}
\end{Shaded}

\hypertarget{income-groups}{%
\section{Income Groups}\label{income-groups}}

The purpose of this function is to calculate income distribution
statistics, as well as the rates of PV ownership by income decile, which
are used in other parts of the model. We begin by sorting
\texttt{\_Income\_p}, the vector containing income net of electricity
cost, in ascending order. Next, we set the cut-off points for income
deciles and percentiles using the vectors \texttt{\_Positions\_p} and
\texttt{\_Positions\_d}, which are calculated during the initialisation
phase (described below) based on the number of agents in the model and
give the number of agents in each decile/percentile. Next we iterate
over all deciles and percentiles, assinging to each agent their income
decile/percentile based on their current income net of electricity cost.

The function returns \texttt{\_IncomePercentiles} and
\texttt{\_IncomeDeciles}, two vectors of length \emph{N}, in which each
element gives the income percentile/decile of the respective agent.

\begin{Shaded}
\begin{Highlighting}[]
\KeywordTok{def}\NormalTok{ IncomeGroups(\_Income\_p,\_Positions\_p,\_Percentiles,\_Positions\_d,\_Deciles,\_IncomePercentiles,\_IncomeDeciles):}
    \CommentTok{\#Sort income net of electricity cost in ascending order}
\NormalTok{    income\_psorted}\OperatorTok{=}\NormalTok{np.sort(\_Income\_p)}
    \CommentTok{\#Based on sorted income and number of agents in each decile/percentile, set cut{-}off points}
\NormalTok{    \_Percentiles[}\DecValTok{0}\NormalTok{:}\DecValTok{99}\NormalTok{]}\OperatorTok{=}\NormalTok{income\_psorted[\_Positions\_p]}
\NormalTok{    \_Percentiles[}\DecValTok{99}\NormalTok{]}\OperatorTok{=}\NormalTok{income\_psorted[((income\_psorted.size)}\OperatorTok{{-}}\DecValTok{1}\NormalTok{)]}
\NormalTok{    \_Deciles[}\DecValTok{0}\NormalTok{:}\DecValTok{9}\NormalTok{]}\OperatorTok{=}\NormalTok{income\_psorted[\_Positions\_d]}
\NormalTok{    \_Deciles[}\DecValTok{9}\NormalTok{]}\OperatorTok{=}\NormalTok{income\_psorted[((income\_psorted.size)}\OperatorTok{{-}}\DecValTok{1}\NormalTok{)]}
    \CommentTok{\#Iterate over deciles}
    \ControlFlowTok{for}\NormalTok{ i }\KeywordTok{in} \BuiltInTok{range}\NormalTok{(}\DecValTok{10}\NormalTok{):}
        \CommentTok{\#Determine which agents belong to decile i}
        \ControlFlowTok{if}\NormalTok{ i}\OperatorTok{==}\DecValTok{0}\NormalTok{:}
\NormalTok{            members}\OperatorTok{=}\NormalTok{np.where(\_Income\_p}\OperatorTok{\textless{}=}\NormalTok{\_Deciles[i])}
        \ControlFlowTok{else}\NormalTok{:}
\NormalTok{            members}\OperatorTok{=}\NormalTok{np.where((\_Income\_p}\OperatorTok{\textless{}=}\NormalTok{\_Deciles[i]) }\OperatorTok{\&}\NormalTok{ (\_Income\_p}\OperatorTok{\textgreater{}}\NormalTok{\_Deciles[(i}\OperatorTok{{-}}\DecValTok{1}\NormalTok{)]))}
        \CommentTok{\#Set decile of all members to i}
\NormalTok{        \_IncomeDeciles[members]}\OperatorTok{=}\NormalTok{i}
    \CommentTok{\#Iterate over percentiles}
    \ControlFlowTok{for}\NormalTok{ i }\KeywordTok{in} \BuiltInTok{range}\NormalTok{(}\DecValTok{100}\NormalTok{):}
        \CommentTok{\#Determine which agents belong to percentile i}
        \ControlFlowTok{if}\NormalTok{ i}\OperatorTok{==}\DecValTok{0}\NormalTok{:}
\NormalTok{            members}\OperatorTok{=}\NormalTok{np.where(\_Income\_p}\OperatorTok{\textless{}=}\NormalTok{\_Percentiles[i])}
        \ControlFlowTok{else}\NormalTok{:}
\NormalTok{            members}\OperatorTok{=}\NormalTok{np.where((\_Income\_p}\OperatorTok{\textless{}=}\NormalTok{\_Percentiles[i]) }\OperatorTok{\&}\NormalTok{ (\_Income\_p}\OperatorTok{\textgreater{}}\NormalTok{\_Percentiles[(i}\OperatorTok{{-}}\DecValTok{1}\NormalTok{)]))}
        \CommentTok{\#Set percentile of all members to i}
\NormalTok{        \_IncomePercentiles[members]}\OperatorTok{=}\NormalTok{i}
    \ControlFlowTok{return}\NormalTok{[\_IncomePercentiles,\_IncomeDeciles]}
\end{Highlighting}
\end{Shaded}

\hypertarget{consumption-and-saving}{%
\section{Consumption and saving}\label{consumption-and-saving}}

This function determines agents' consumption and saving. In addition,
agents with positive outstanding debt make debt service payments. The
main purpose of this is to update agents' \texttt{\_Liquidity} ,
i.e.~their stock of liquid assets (money) which can be used to buy PV.

The function begins by setting a consumption propensity out of income
net of electricity cost for each agent based on their current income
percentile. From this, their desired consumption is determined,
including a persistence component. Income, electricity cost and desired
consumption are then used to calculate the desired saving of each agent.
If an agent cannot finance their desired consumption and electricity
cost using current income and accumulated liquid assets, their actual
consumption is implicitly curtailed such that their liquidity does not
become negative (since we assume that households cannot borrow for
consumption but only to acquire PV).

Next, agents with positive debt make debt service payments, which are
divided into principal payments (which reduce the stock of outstanding
debt) and interest payments. Any interest payments which agents cannot
afford are added to their balance of outstanding debt and their debt
service payments for the following periods are updated accordingly.
Agents whose debt has become zero in the current period (i.e.~they have
paid off their full loan) have their interest rate and debt service
payments set to zero. \texttt{\_CumulativeProfit}, which keeps track of
the profit/cost saving which PV owners have derived from their solar
panel is updated using interest payments made.

Finally, if an agent has positive saving after debt service in the
current period, their liquidity is updated by a fraction of that
increment. This fraction is based on the agent's current income decile
and reflects the share of financial wealth held as liquid assets by
income group.

The function returns agents' updated current liquidity, debt, debt
service payments, loan interest rates and \emph{desired} consumption
(the latter is used in the next period to calculate the persistence
component of desired consumption).

\begin{Shaded}
\begin{Highlighting}[]
\KeywordTok{def}\NormalTok{ ConsumptionSaving(\_Income\_p,\_Propensities,\_IncomePercentiles,\_Consumption,\_Income,\_Powercost,\_Liquidity,\_Debt,\_DebtService,\_LoanRate,\_PanelAge,\_CumulativeProfit,\_LiquidityShares,\_Params,\_period):}
    \CommentTok{\#Determine agents\textquotesingle{} consumption propensities out of income net of electricity cost based on their current income percentile}
\NormalTok{    consshares}\OperatorTok{=}\NormalTok{\_Propensities[\_IncomePercentiles.astype(}\BuiltInTok{int}\NormalTok{)]}
    \CommentTok{\#Determine desired consumption}
    \ControlFlowTok{if}\NormalTok{ \_period}\OperatorTok{==}\DecValTok{0}\NormalTok{:}
\NormalTok{        \_Consumption[:]}\OperatorTok{=}\NormalTok{consshares}\OperatorTok{*}\NormalTok{\_Income\_p}
    \ControlFlowTok{else}\NormalTok{:}
        \CommentTok{\#If simulation period\textgreater{}0, take into account persistence in desired consumption}
\NormalTok{        \_Consumption[:]}\OperatorTok{=}\NormalTok{\_Params[}\StringTok{"PersistenceConsumption"}\NormalTok{].values[}\DecValTok{0}\NormalTok{]}\OperatorTok{*}\NormalTok{\_Consumption[:]}\OperatorTok{+}\NormalTok{(}\DecValTok{1}\OperatorTok{{-}}\NormalTok{\_Params[}\StringTok{"PersistenceConsumption"}\NormalTok{].values[}\DecValTok{0}\NormalTok{])}\OperatorTok{*}\NormalTok{consshares}\OperatorTok{*}\NormalTok{\_Income\_p}
    \CommentTok{\#Agents\textquotesingle{} desired saving is gross income minus (desired) consumption minus electricity cost}
\NormalTok{    saving}\OperatorTok{=}\NormalTok{\_Income}\OperatorTok{{-}}\NormalTok{\_Consumption}\OperatorTok{{-}}\NormalTok{\_Powercost}
    \CommentTok{\#Save pre{-}consumption liquidity (stock of money) of each agent}
\NormalTok{    Liquidity\_p}\OperatorTok{=}\NormalTok{\_Liquidity.copy()}
    \CommentTok{\#If liquidity and income are greater than consumption plus electricity cost, update liquidity using saving (which may be negative!)}
\NormalTok{    \_Liquidity[(Liquidity\_p}\OperatorTok{+}\NormalTok{\_Income)}\OperatorTok{\textgreater{}=}\NormalTok{(\_Consumption}\OperatorTok{+}\NormalTok{\_Powercost)]}\OperatorTok{=}\NormalTok{\_Liquidity[(Liquidity\_p}\OperatorTok{+}\NormalTok{\_Income)}\OperatorTok{\textgreater{}=}\NormalTok{(\_Consumption}\OperatorTok{+}\NormalTok{\_Powercost)]}\OperatorTok{+}\NormalTok{saving[(Liquidity\_p}\OperatorTok{+}\NormalTok{\_Income)}\OperatorTok{\textgreater{}=}\NormalTok{(\_Consumption}\OperatorTok{+}\NormalTok{\_Powercost)]}
    \CommentTok{\#Otherwise, liquidity is set to 0; consumption is implicitly reduced to ensure liquidity does not become negative}
\NormalTok{    \_Liquidity[(Liquidity\_p}\OperatorTok{+}\NormalTok{\_Income)}\OperatorTok{\textless{}}\NormalTok{(\_Consumption}\OperatorTok{+}\NormalTok{\_Powercost)]}\OperatorTok{=}\DecValTok{0}
    \CommentTok{\#Determine debt service for agents with positive debt}
\NormalTok{    \_DebtService[\_Debt}\OperatorTok{\textgreater{}}\DecValTok{0}\NormalTok{]}\OperatorTok{=}\NormalTok{np.minimum(\_DebtService[\_Debt}\OperatorTok{\textgreater{}}\DecValTok{0}\NormalTok{],\_Debt[\_Debt}\OperatorTok{\textgreater{}}\DecValTok{0}\NormalTok{])}
    \CommentTok{\#Decompose debt service into interest and principal component}
\NormalTok{    interest}\OperatorTok{=}\NormalTok{\_LoanRate}\OperatorTok{*}\NormalTok{\_Debt}
\NormalTok{    principal}\OperatorTok{=}\NormalTok{\_DebtService}\OperatorTok{{-}}\NormalTok{interest}
    \CommentTok{\#Update remaining term of loan}
\NormalTok{    term}\OperatorTok{=}\NormalTok{np.maximum(}\DecValTok{1}\NormalTok{,\_Params[}\StringTok{"PVMaxAge"}\NormalTok{].values[}\DecValTok{0}\NormalTok{]}\OperatorTok{{-}}\NormalTok{\_PanelAge)}
    \CommentTok{\#Principal is paid if agent can afford to do so; debt is reduced}
\NormalTok{    \_Debt[\_Liquidity}\OperatorTok{\textgreater{}=}\NormalTok{\_DebtService]}\OperatorTok{=}\NormalTok{\_Debt[\_Liquidity}\OperatorTok{\textgreater{}=}\NormalTok{\_DebtService]}\OperatorTok{{-}}\NormalTok{principal[\_Liquidity}\OperatorTok{\textgreater{}=}\NormalTok{\_DebtService]}
    \CommentTok{\#Agents who cannot afford full principal pay as much as they can}
\NormalTok{    \_Debt[(\_Liquidity}\OperatorTok{\textgreater{}=}\NormalTok{interest) }\OperatorTok{\&}\NormalTok{ (\_Liquidity}\OperatorTok{\textless{}}\NormalTok{\_DebtService)]}\OperatorTok{=}\NormalTok{\_Debt[(\_Liquidity}\OperatorTok{\textgreater{}=}\NormalTok{interest) }\OperatorTok{\&}\NormalTok{ (\_Liquidity}\OperatorTok{\textless{}}\NormalTok{\_DebtService)]}\OperatorTok{{-}}\NormalTok{\_Liquidity[(\_Liquidity}\OperatorTok{\textgreater{}=}\NormalTok{interest) }\OperatorTok{\&}\NormalTok{ (\_Liquidity}\OperatorTok{\textless{}}\NormalTok{\_DebtService)]}
    \CommentTok{\#Any unpaid interest is added to outstanding debt}
\NormalTok{    \_Debt[\_Liquidity}\OperatorTok{\textless{}}\NormalTok{interest]}\OperatorTok{=}\NormalTok{\_Debt[\_Liquidity}\OperatorTok{\textless{}}\NormalTok{interest]}\OperatorTok{+}\NormalTok{interest[\_Liquidity}\OperatorTok{\textless{}}\NormalTok{interest]}\OperatorTok{{-}}\NormalTok{\_Liquidity[\_Liquidity}\OperatorTok{\textless{}}\NormalTok{interest]}
    \CommentTok{\#Update cumulative profit of PV owners by interested paid on loans to acquire PV}
\NormalTok{    interest[\_Liquidity}\OperatorTok{\textless{}}\NormalTok{interest]}\OperatorTok{=}\NormalTok{interest[\_Liquidity}\OperatorTok{\textless{}}\NormalTok{interest]}\OperatorTok{{-}}\NormalTok{\_Liquidity[\_Liquidity}\OperatorTok{\textless{}}\NormalTok{interest]}
\NormalTok{    \_CumulativeProfit}\OperatorTok{=}\NormalTok{\_CumulativeProfit}\OperatorTok{{-}}\NormalTok{interest}
    \CommentTok{\#Find agents who did not fully service debt}
\NormalTok{    adjustservice}\OperatorTok{=}\NormalTok{np.where(\_Liquidity}\OperatorTok{\textless{}}\NormalTok{\_DebtService)}
     \CommentTok{\#Update liquidity based on debt service made}
\NormalTok{    Liquidity\_pd}\OperatorTok{=}\NormalTok{\_Liquidity.copy()}
\NormalTok{    \_Liquidity[Liquidity\_pd}\OperatorTok{\textless{}}\NormalTok{\_DebtService]}\OperatorTok{=}\DecValTok{0}
\NormalTok{    \_Liquidity[Liquidity\_pd}\OperatorTok{\textgreater{}=}\NormalTok{\_DebtService]}\OperatorTok{=}\NormalTok{\_Liquidity[Liquidity\_pd}\OperatorTok{\textgreater{}=}\NormalTok{\_DebtService]}\OperatorTok{{-}}\NormalTok{\_DebtService[Liquidity\_pd}\OperatorTok{\textgreater{}=}\NormalTok{\_DebtService]}
    \CommentTok{\#Adjust debt service of agents who did not pay full debt service}
\NormalTok{    \_DebtService[adjustservice]}\OperatorTok{=}\NormalTok{\_Debt[adjustservice]}\OperatorTok{*}\NormalTok{(\_LoanRate[adjustservice]}\OperatorTok{*}\NormalTok{np.power((}\DecValTok{1}\OperatorTok{+}\NormalTok{\_LoanRate[adjustservice]),term[adjustservice]))}\OperatorTok{/}\NormalTok{(np.power((}\DecValTok{1}\OperatorTok{+}\NormalTok{\_LoanRate[adjustservice]),term[adjustservice])}\OperatorTok{{-}}\DecValTok{1}\NormalTok{)}
     \CommentTok{\#Set loan rate, debt service and debt of agents who have paid off their loans to 0 (\textless{}=0) is used here since rounding errors can lead to slightly negative debt}
\NormalTok{    \_LoanRate[\_Debt}\OperatorTok{\textless{}=}\DecValTok{0}\NormalTok{]}\OperatorTok{=}\DecValTok{0}
\NormalTok{    \_DebtService[\_Debt}\OperatorTok{\textless{}=}\DecValTok{0}\NormalTok{]}\OperatorTok{=}\DecValTok{0}
\NormalTok{    \_Debt[\_Debt}\OperatorTok{\textless{}=}\DecValTok{0}\NormalTok{]}\OperatorTok{=}\DecValTok{0}
    \CommentTok{\#Update liquidity based on shares of savings going into liquid assets by income percentile}
\NormalTok{    shares}\OperatorTok{=}\NormalTok{\_LiquidityShares[\_IncomePercentiles.astype(}\BuiltInTok{int}\NormalTok{)]}
\NormalTok{    \_Liquidity[\_Liquidity}\OperatorTok{\textgreater{}}\NormalTok{Liquidity\_p]}\OperatorTok{=}\NormalTok{Liquidity\_p[\_Liquidity}\OperatorTok{\textgreater{}}\NormalTok{Liquidity\_p]}\OperatorTok{+}\NormalTok{shares[\_Liquidity}\OperatorTok{\textgreater{}}\NormalTok{Liquidity\_p]}\OperatorTok{*}\NormalTok{(\_Liquidity[\_Liquidity}\OperatorTok{\textgreater{}}\NormalTok{Liquidity\_p]}\OperatorTok{{-}}\NormalTok{Liquidity\_p[\_Liquidity}\OperatorTok{\textgreater{}}\NormalTok{Liquidity\_p])}
\end{Highlighting}
\end{Shaded}

\hypertarget{augment-panel-age}{%
\section{Augment panel age}\label{augment-panel-age}}

This short function augments the age of all solar panels in the model by
one period. If the panel of an agent has reached its maximum age, the
value of \texttt{\_PV} is set to zero for that agent, as is the value of
\texttt{\_PanelAge} and \texttt{\_PanelFIT} which gives the FIT
received. The agent is then no longer a PV owner and will face the
adoption decision again.

\begin{Shaded}
\begin{Highlighting}[]
\KeywordTok{def}\NormalTok{ AugmentPanelAge(\_PV,\_PanelAge,\_PanelFIT,\_Params):}
\NormalTok{    \_PanelAge[\_PV}\OperatorTok{==}\DecValTok{1}\NormalTok{]}\OperatorTok{=}\NormalTok{\_PanelAge[\_PV}\OperatorTok{==}\DecValTok{1}\NormalTok{]}\OperatorTok{+}\DecValTok{1}
\NormalTok{    \_PV[\_PanelAge}\OperatorTok{\textgreater{}}\NormalTok{\_Params[}\StringTok{"PVMaxAge"}\NormalTok{].values[}\DecValTok{0}\NormalTok{]]}\OperatorTok{=}\DecValTok{0}
\NormalTok{    \_PanelFIT[\_PanelAge}\OperatorTok{\textgreater{}}\NormalTok{\_Params[}\StringTok{"PVMaxAge"}\NormalTok{].values[}\DecValTok{0}\NormalTok{]]}\OperatorTok{=}\DecValTok{0}
\NormalTok{    \_PanelAge[\_PanelAge}\OperatorTok{\textgreater{}}\NormalTok{\_Params[}\StringTok{"PVMaxAge"}\NormalTok{].values[}\DecValTok{0}\NormalTok{]]}\OperatorTok{=}\DecValTok{0}
    \ControlFlowTok{return}\NormalTok{[\_PanelAge,\_PV,\_PanelFIT]}
\end{Highlighting}
\end{Shaded}

\hypertarget{expectations-formation}{%
\section{Expectations formation}\label{expectations-formation}}

This function is used to generate a time-series of expected future
electricity prices for each agents. These are needed below when agents
decide whether or not to adopt PV.

Expectations formation of our bounded rational agents takes place using
the well-established recursive least squares algorithm. Each agent has
their own ``internal'' AR(1) model of the electricity price, the
coefficients of which they update in every period using the latest
observation of the actual electricity price in the model. Depending on
the parameter setting, each individual agents may use a specific gain
parameter to update their estimations, in which case the expected
electricity prices differ across agents. In particular, agents with a
high gain parameter will tend to update their estimated coefficients
more strongly in reaction to new information, while agents with a low
gain parameter will react less.

The first part of the function updates the internal AR(1) models of the
agents, while the second projects the models to generate time-series of
a length equal to the maximum lifespan of a solar panel. The function
then returns the coefficients variance-covariance matrices of the
updated AR(1) models and the time-series of expected prices.

\begin{Shaded}
\begin{Highlighting}[]
\KeywordTok{def}\NormalTok{ FormExpectation(\_ARPars,\_ARMats,\_Price,\_Gain,\_ExpectedPrice,\_period):}
    \CommentTok{\#Generate vector of independent variables {-}{-}\textgreater{} constant=1 and current change in price}
\NormalTok{    independent}\OperatorTok{=}\NormalTok{np.array([[}\DecValTok{1}\NormalTok{],[(\_Price[(\_period}\OperatorTok{+}\DecValTok{1}\NormalTok{)]}\OperatorTok{{-}}\NormalTok{\_Price[\_period])]])}
    \CommentTok{\#Update variance{-}covariance matrices following recursive least squares algorithm}
\NormalTok{    mat1}\OperatorTok{=}\NormalTok{np.dot(independent,independent.transpose())}
\NormalTok{    ARMatsNew}\OperatorTok{=}\NormalTok{\_ARMats}\OperatorTok{+}\NormalTok{\_Gain[:,}\VariableTok{None}\NormalTok{,}\VariableTok{None}\NormalTok{]}\OperatorTok{*}\NormalTok{(mat1}\OperatorTok{{-}}\NormalTok{\_ARMats)}
    \CommentTok{\#Test for singularity of updated matrices}
\NormalTok{    test}\OperatorTok{=}\NormalTok{np.linalg.det(ARMatsNew)}
    \ControlFlowTok{if}\NormalTok{(}\BuiltInTok{sum}\NormalTok{(test}\OperatorTok{==}\DecValTok{0}\NormalTok{)}\OperatorTok{\textgreater{}}\DecValTok{0}\NormalTok{):}
        \BuiltInTok{print}\NormalTok{(}\StringTok{\textquotesingle{}One or more matrices are singular!\textquotesingle{}}\NormalTok{)}
    \CommentTok{\#If updated matrix of an agent is singular, use old matrix}
\NormalTok{    \_ARMats[np.where(test}\OperatorTok{!=}\DecValTok{0}\NormalTok{)]}\OperatorTok{=}\NormalTok{ARMatsNew[np.where(test}\OperatorTok{!=}\DecValTok{0}\NormalTok{)]}
    \CommentTok{\#Update coefficient estimates of agents\textquotesingle{} AR(1) models}
\NormalTok{    transposes}\OperatorTok{=}\NormalTok{\_ARPars.swapaxes(}\DecValTok{1}\NormalTok{,}\DecValTok{2}\NormalTok{)}
\NormalTok{    result1}\OperatorTok{=}\NormalTok{np.dot(transposes,independent)}
\NormalTok{    dependent}\OperatorTok{=}\NormalTok{\_Price[(\_period}\OperatorTok{+}\DecValTok{2}\NormalTok{)]}\OperatorTok{{-}}\NormalTok{\_Price[(\_period}\OperatorTok{+}\DecValTok{1}\NormalTok{)]}
\NormalTok{    result1}\OperatorTok{=}\NormalTok{dependent}\OperatorTok{{-}}\NormalTok{result1}
\NormalTok{    result1}\OperatorTok{=}\NormalTok{independent}\OperatorTok{*}\NormalTok{result1}
\NormalTok{    result1}\OperatorTok{=}\NormalTok{np.linalg.solve(\_ARMats,result1)}
\NormalTok{    \_ARPars}\OperatorTok{=}\NormalTok{\_ARPars}\OperatorTok{+}\NormalTok{\_Gain[:,}\VariableTok{None}\NormalTok{,}\VariableTok{None}\NormalTok{]}\OperatorTok{*}\NormalTok{result1}
\NormalTok{    transposes1}\OperatorTok{=}\NormalTok{\_ARPars[:,}\DecValTok{0}\NormalTok{].transpose()}
\NormalTok{    transposes2}\OperatorTok{=}\NormalTok{\_ARPars[:,}\DecValTok{1}\NormalTok{].transpose()}
    \CommentTok{\#Generate time{-}series of expected electricity price for each agent}
\NormalTok{    \_ExpectedPrice[:,}\DecValTok{0}\NormalTok{]}\OperatorTok{=}\NormalTok{(\_Price[(\_period}\OperatorTok{+}\DecValTok{2}\NormalTok{)]}\OperatorTok{+}\NormalTok{\_ARPars[:,}\DecValTok{0}\NormalTok{]}\OperatorTok{+}\NormalTok{\_ARPars[:,}\DecValTok{1}\NormalTok{]}\OperatorTok{*}\NormalTok{(\_Price[(\_period}\OperatorTok{+}\DecValTok{2}\NormalTok{)]}\OperatorTok{{-}}\NormalTok{\_Price[(\_period}\OperatorTok{+}\DecValTok{1}\NormalTok{)])).transpose()}
\NormalTok{    \_ExpectedPrice[:,}\DecValTok{1}\NormalTok{]}\OperatorTok{=}\NormalTok{\_ExpectedPrice[:,}\DecValTok{0}\NormalTok{]}\OperatorTok{+}\NormalTok{transposes1}\OperatorTok{+}\NormalTok{transposes2}\OperatorTok{*}\NormalTok{(\_ExpectedPrice[:,}\DecValTok{0}\NormalTok{]}\OperatorTok{{-}}\NormalTok{\_Price[(\_period}\OperatorTok{+}\DecValTok{2}\NormalTok{)])}
    \ControlFlowTok{for}\NormalTok{ t }\KeywordTok{in} \BuiltInTok{range}\NormalTok{(}\DecValTok{2}\NormalTok{,}\BuiltInTok{len}\NormalTok{(\_ExpectedPrice[}\DecValTok{0}\NormalTok{])):}
\NormalTok{        \_ExpectedPrice[:,t]}\OperatorTok{=}\NormalTok{\_ExpectedPrice[:,(t}\OperatorTok{{-}}\DecValTok{1}\NormalTok{)]}\OperatorTok{+}\NormalTok{transposes1}\OperatorTok{+}\NormalTok{transposes2}\OperatorTok{*}\NormalTok{(\_ExpectedPrice[:,(t}\OperatorTok{{-}}\DecValTok{1}\NormalTok{)]}\OperatorTok{{-}}\NormalTok{\_ExpectedPrice[:,(t}\OperatorTok{{-}}\DecValTok{2}\NormalTok{)])}
    \ControlFlowTok{return}\NormalTok{[\_ARMats,\_ARPars,\_ExpectedPrice]}
\end{Highlighting}
\end{Shaded}

\hypertarget{adoption-decision}{%
\section{Adoption Decision}\label{adoption-decision}}

In this function, agents who do not yet own PV decide whether or not to
purchase a panel.

The function first checks which agents would need a loan in order to
adopt PV (namely those who do not have sufficient liquid assets to buy
one outright). Based on the size of the loan needed, agents may be
credit rationed if the resulting loan to value ratio exceeds the maximum
permitted level (note that in the current implementation,
\texttt{\_Params{[}"MaxLTV"{]}=1} such that this rationing mechanism is
effectively switched off).

Next, the bank proposes a loan interest rate to every agent with
positive borrowing needs based on the size of the required loan. From
this, in turn, we calculate the size of the implied debt service
payment. If the prospective debt service payment of an agent exceeds a
certain fraction (\texttt{\_Params{[}"MaxDTI"{]}}) of their current
income, the agent is credit rationed. Since the model only includes a
single type of solar panel with a given installation cost, credit
rationed agents cannot adopt (i.e.~they cannot, for instance, buy a
smaller panel with a smaller loan).

The function then calculates, for each agent, a sum of expected
discounted revenue from owning PV over the lifetime of a panel, based on
agents' expected future electricity prices, the value of the feed-in
tariff for panels installed in the current year, agents' future
electricity consumption (which, as outlined above, grows at a fixed rate
every year), as well as a sum of discounted future loan costs based on
the size of the needed loan and the interest rate proposed by the bank.
As in the case of the function \texttt{IncomePowercost} outlined above,
in calculating the expected cost-saving/revenue from owning PV, agents
are divided into groups depending on whether their future electricity
consumption is larger or smaller than the capacity of the solar panel
and whether the current FIT is larger or smaller than the expected
electricity price. Based on the cumulative discounted expected revenue
and loan cost and the one-off installation cost, an expected profit from
owning PV is calculated.

This expected profit, in turn, feeds into the expected utility from
adopting PV. In addition to profit, this utility also contains a
component related to social influence (based on the difference between
the share of other agents in a given agent's income decile who have
already adopted PV and those who have not) and one related to
environmental attitude.

Agents for whom expected utility is positive (and for whom PV is
technically feasible and who are not credit-rationed) purchase a solar
panel and their liquidity, debt, debt service payments and loan rates
are updated accordingly.

\begin{Shaded}
\begin{Highlighting}[]
\KeywordTok{def}\NormalTok{ AdoptionDecision(\_PV,\_Feasible,\_Liquidity,\_Rationed,\_Debt,\_Income,\_Income\_f,\_Powercost\_f,\_Trend,\_ExpectedPrice,\_Discount,\_Revenue,\_LoanCost,\_PVOwnershipDeciles,\_IncomeDeciles,\_LoanRate,\_DebtService,\_Influence,\_Attitude,\_Adopted,\_CumulativeProfit,\_FIT,\_year,\_PanelFIT,\_Baserate,\_Subsidy1,\_Subsidy2,\_CreditPolicy1,\_CreditPolicy2,\_Params):}
    \CommentTok{\#Reset vector of dummies indicating whether agents are credit{-}rationed}
\NormalTok{    \_Rationed[:]}\OperatorTok{=}\DecValTok{0}
    \CommentTok{\#Calculate loan needed to purchase PV}
\NormalTok{    loanneeded}\OperatorTok{=}\NormalTok{(}\DecValTok{1}\OperatorTok{{-}}\NormalTok{\_Subsidy1)}\OperatorTok{*}\NormalTok{\_Params[}\StringTok{"PanelCost"}\NormalTok{].values[}\DecValTok{0}\NormalTok{]}\OperatorTok{{-}}\NormalTok{\_Liquidity}
    \CommentTok{\#Agents who already have PV do not need a loan}
\NormalTok{    loanneeded[\_PV}\OperatorTok{==}\DecValTok{1}\NormalTok{]}\OperatorTok{=}\DecValTok{0}
    \CommentTok{\#Agents who can buy the panel outright do not need a loan}
\NormalTok{    loanneeded[(\_PV}\OperatorTok{==}\DecValTok{0}\NormalTok{) }\OperatorTok{\&}\NormalTok{ (\_Liquidity}\OperatorTok{\textgreater{}=}\NormalTok{((}\DecValTok{1}\OperatorTok{{-}}\NormalTok{\_Subsidy1)}\OperatorTok{*}\NormalTok{\_Params[}\StringTok{"PanelCost"}\NormalTok{].values[}\DecValTok{0}\NormalTok{]))]}\OperatorTok{=}\DecValTok{0}
    \CommentTok{\#Agents for whom PV is not technically feasible do not need a loan}
\NormalTok{    loanneeded[\_Feasible}\OperatorTok{==}\DecValTok{0}\NormalTok{]}\OperatorTok{=}\DecValTok{0}
    \CommentTok{\#Check whether needed loan exceeds maximum loan to value ratio}
\NormalTok{    \_Rationed[(loanneeded}\OperatorTok{+}\NormalTok{\_Debt)}\OperatorTok{\textgreater{}}\NormalTok{((}\DecValTok{1}\OperatorTok{{-}}\NormalTok{\_Subsidy1)}\OperatorTok{*}\NormalTok{\_Params[}\StringTok{"PanelCost"}\NormalTok{].values[}\DecValTok{0}\NormalTok{]}\OperatorTok{*}\NormalTok{(\_Params[}\StringTok{"MaxLTV"}\NormalTok{].values[}\DecValTok{0}\NormalTok{]}\OperatorTok{+}\NormalTok{\_CreditPolicy1))]}\OperatorTok{=}\DecValTok{1}
    \CommentTok{\#Calculate interest rate proposed by the bank based on size of loan needed}
\NormalTok{    proposedrate}\OperatorTok{=}\NormalTok{\_Baserate[}\BuiltInTok{str}\NormalTok{(\_year)].values[}\DecValTok{0}\NormalTok{]}\OperatorTok{+}\NormalTok{\_Params[}\StringTok{"InterestMarkup"}\NormalTok{].values[}\DecValTok{0}\NormalTok{]}\OperatorTok{*}\NormalTok{(}\DecValTok{1}\OperatorTok{+}\NormalTok{loanneeded}\OperatorTok{/}\NormalTok{((}\DecValTok{1}\OperatorTok{{-}}\NormalTok{\_Subsidy1)}\OperatorTok{*}\NormalTok{\_Params[}\StringTok{"PanelCost"}\NormalTok{].values[}\DecValTok{0}\NormalTok{]}\OperatorTok{*}\NormalTok{\_Params[}\StringTok{"MaxLTV"}\NormalTok{].values[}\DecValTok{0}\NormalTok{]))}
    \CommentTok{\#Calculate size of debt service payments}
\NormalTok{    payment}\OperatorTok{=}\NormalTok{loanneeded}\OperatorTok{*}\NormalTok{(proposedrate}\OperatorTok{*}\NormalTok{(np.power(}\DecValTok{1}\OperatorTok{+}\NormalTok{proposedrate,\_Params[}\StringTok{"PVMaxAge"}\NormalTok{].values[}\DecValTok{0}\NormalTok{])))}\OperatorTok{/}\NormalTok{((np.power(}\DecValTok{1}\OperatorTok{+}\NormalTok{proposedrate,\_Params[}\StringTok{"PVMaxAge"}\NormalTok{].values[}\DecValTok{0}\NormalTok{]))}\OperatorTok{{-}}\DecValTok{1}\NormalTok{)}
    \CommentTok{\#Check whether implied debt service to income ratio exceeds maximum permitted}
\NormalTok{    \_Rationed[(payment}\OperatorTok{/}\NormalTok{\_Income)}\OperatorTok{\textgreater{}}\NormalTok{(\_Params[}\StringTok{"MaxDTI"}\NormalTok{].values[}\DecValTok{0}\NormalTok{]}\OperatorTok{+}\NormalTok{\_CreditPolicy2)]}\OperatorTok{=}\DecValTok{1}
    \CommentTok{\#Agents who do not need a loan are not rationed}
\NormalTok{    \_Rationed[loanneeded}\OperatorTok{==}\DecValTok{0}\NormalTok{]}\OperatorTok{=}\DecValTok{0}
    \CommentTok{\#Agents for whom PV is not technically feasible are not rationed}
\NormalTok{    \_Rationed[\_Feasible}\OperatorTok{==}\DecValTok{0}\NormalTok{]}\OperatorTok{=}\DecValTok{0}
    \CommentTok{\#Agents who already own PV are not rationed}
\NormalTok{    \_Rationed[\_PV}\OperatorTok{==}\DecValTok{1}\NormalTok{]}\OperatorTok{=}\DecValTok{0}
    \CommentTok{\#Initialise expected revenue from PV and cost of loan}
\NormalTok{    \_Revenue[:]}\OperatorTok{=}\DecValTok{0}
\NormalTok{    \_LoanCost[:]}\OperatorTok{=}\DecValTok{0}
    \CommentTok{\#Initialise cost saving from PV}
\NormalTok{    costsaving}\OperatorTok{=}\NormalTok{\_Revenue.copy()}
    \CommentTok{\#Initial remaining loan balance is equal to size of loan needed}
\NormalTok{    balanceremaining}\OperatorTok{=}\NormalTok{loanneeded.copy()}
    \CommentTok{\#Iterate over lifespan of a solar panel}
    \ControlFlowTok{for}\NormalTok{ t }\KeywordTok{in} \BuiltInTok{range}\NormalTok{(\_Params[}\StringTok{"PVMaxAge"}\NormalTok{].values[}\DecValTok{0}\NormalTok{]):}
\NormalTok{        costsaving[:]}\OperatorTok{=}\DecValTok{0}
        \CommentTok{\#Agents whose expected electricity consumption exceeds the capacity of the panel and for whom the current FIT is smaller than the expected electricity price}
\NormalTok{        group1}\OperatorTok{=}\NormalTok{np.where((\_ExpectedPrice[:,t]}\OperatorTok{\textgreater{}=}\NormalTok{(\_FIT[}\BuiltInTok{str}\NormalTok{(\_year)].values[}\DecValTok{0}\NormalTok{]}\OperatorTok{+}\NormalTok{\_Subsidy2)) }\OperatorTok{\&}\NormalTok{ ((\_Powercost\_f}\OperatorTok{*}\NormalTok{np.power((}\DecValTok{1}\OperatorTok{+}\NormalTok{\_Trend),t))}\OperatorTok{\textgreater{}=}\NormalTok{\_Params[}\StringTok{"PVCapacity"}\NormalTok{].values[}\DecValTok{0}\NormalTok{]))}
        \CommentTok{\#Agents whose expected electricity consumption is smaller than the capacity of the panel and for whom the current FIT is smaller than the expected electricity price}
\NormalTok{        group2}\OperatorTok{=}\NormalTok{np.where((\_ExpectedPrice[:,t]}\OperatorTok{\textgreater{}=}\NormalTok{(\_FIT[}\BuiltInTok{str}\NormalTok{(\_year)].values[}\DecValTok{0}\NormalTok{]}\OperatorTok{+}\NormalTok{\_Subsidy2)) }\OperatorTok{\&}\NormalTok{ ((\_Powercost\_f}\OperatorTok{*}\NormalTok{np.power((}\DecValTok{1}\OperatorTok{+}\NormalTok{\_Trend),t))}\OperatorTok{\textless{}}\NormalTok{\_Params[}\StringTok{"PVCapacity"}\NormalTok{].values[}\DecValTok{0}\NormalTok{]))}
        \CommentTok{\#Agents whose expected electricity consumption exceeds the capacity of the panel and for whom the current FIT is larger than the expected electricity price}
\NormalTok{        group3}\OperatorTok{=}\NormalTok{np.where((\_ExpectedPrice[:,t]}\OperatorTok{\textless{}}\NormalTok{(\_FIT[}\BuiltInTok{str}\NormalTok{(\_year)].values[}\DecValTok{0}\NormalTok{]}\OperatorTok{+}\NormalTok{\_Subsidy2)) }\OperatorTok{\&}\NormalTok{ ((\_Powercost\_f}\OperatorTok{*}\NormalTok{np.power((}\DecValTok{1}\OperatorTok{+}\NormalTok{\_Trend),t))}\OperatorTok{\textgreater{}=}\NormalTok{\_Params[}\StringTok{"PVCapacity"}\NormalTok{].values[}\DecValTok{0}\NormalTok{]))}
        \CommentTok{\#Agents whose expected electricity consumption is smaller than the capacity of the panel and for whom the current FIT is larger than the expected electricity price}
\NormalTok{        group4}\OperatorTok{=}\NormalTok{np.where((\_ExpectedPrice[:,t]}\OperatorTok{\textless{}}\NormalTok{(\_FIT[}\BuiltInTok{str}\NormalTok{(\_year)].values[}\DecValTok{0}\NormalTok{]}\OperatorTok{+}\NormalTok{\_Subsidy2)) }\OperatorTok{\&}\NormalTok{ ((\_Powercost\_f}\OperatorTok{*}\NormalTok{np.power((}\DecValTok{1}\OperatorTok{+}\NormalTok{\_Trend),t))}\OperatorTok{\textless{}}\NormalTok{\_Params[}\StringTok{"PVCapacity"}\NormalTok{].values[}\DecValTok{0}\NormalTok{]))}
        \CommentTok{\#expected cost saving for group 1 is given by share of own consumption times capacity times expected price plus electricity fed into the grid times FIT}
\NormalTok{        costsaving[group1]}\OperatorTok{=}\NormalTok{\_ExpectedPrice[group1,t]}\OperatorTok{*}\NormalTok{\_Params[}\StringTok{"PVCapacity"}\NormalTok{].values[}\DecValTok{0}\NormalTok{]}\OperatorTok{*}\NormalTok{\_Params[}\StringTok{"PVCons"}\NormalTok{].values[}\DecValTok{0}\NormalTok{]}\OperatorTok{+}\NormalTok{(}\DecValTok{1}\OperatorTok{{-}}\NormalTok{\_Params[}\StringTok{"PVCons"}\NormalTok{].values[}\DecValTok{0}\NormalTok{])}\OperatorTok{*}\NormalTok{(\_FIT[}\BuiltInTok{str}\NormalTok{(\_year)].values[}\DecValTok{0}\NormalTok{]}\OperatorTok{+}\NormalTok{\_Subsidy2)}\OperatorTok{*}\NormalTok{\_Params[}\StringTok{"PVCapacity"}\NormalTok{].values[}\DecValTok{0}\NormalTok{]}
        \CommentTok{\#expected cost saving for group 2 is given by minimum between own consumption share times capacity times expected price and expected electricity consumption times expected price, plus the remaining capacity of the panel times FIT}
\NormalTok{        costsaving[group2]}\OperatorTok{=}\NormalTok{\_ExpectedPrice[group2,t]}\OperatorTok{*}\NormalTok{np.minimum(\_Params[}\StringTok{"PVCons"}\NormalTok{].values[}\DecValTok{0}\NormalTok{]}\OperatorTok{*}\NormalTok{\_Params[}\StringTok{"PVCapacity"}\NormalTok{].values[}\DecValTok{0}\NormalTok{],\_Powercost\_f[group2]}\OperatorTok{*}\NormalTok{np.power((}\DecValTok{1}\OperatorTok{+}\NormalTok{\_Trend[group2]),t))}\OperatorTok{+}\NormalTok{(\_FIT[}\BuiltInTok{str}\NormalTok{(\_year)].values[}\DecValTok{0}\NormalTok{]}\OperatorTok{+}\NormalTok{\_Subsidy2)}\OperatorTok{*}\NormalTok{(\_Params[}\StringTok{"PVCapacity"}\NormalTok{].values[}\DecValTok{0}\NormalTok{]}\OperatorTok{{-}}\NormalTok{np.minimum(\_Params[}\StringTok{"PVCons"}\NormalTok{].values[}\DecValTok{0}\NormalTok{]}\OperatorTok{*}\NormalTok{\_Params[}\StringTok{"PVCapacity"}\NormalTok{].values[}\DecValTok{0}\NormalTok{],\_Powercost\_f[group2]}\OperatorTok{*}\NormalTok{np.power((}\DecValTok{1}\OperatorTok{+}\NormalTok{\_Trend[group2]),t)))}
        \CommentTok{\#expected cost saving for groups 3 and 4 is equal to FIT times capacity of the solar panel}
\NormalTok{        costsaving[group3]}\OperatorTok{=}\NormalTok{(\_FIT[}\BuiltInTok{str}\NormalTok{(\_year)].values[}\DecValTok{0}\NormalTok{]}\OperatorTok{+}\NormalTok{\_Subsidy2)}\OperatorTok{*}\NormalTok{\_Params[}\StringTok{"PVCapacity"}\NormalTok{].values[}\DecValTok{0}\NormalTok{]}
\NormalTok{        costsaving[group4]}\OperatorTok{=}\NormalTok{(\_FIT[}\BuiltInTok{str}\NormalTok{(\_year)].values[}\DecValTok{0}\NormalTok{]}\OperatorTok{+}\NormalTok{\_Subsidy2)}\OperatorTok{*}\NormalTok{\_Params[}\StringTok{"PVCapacity"}\NormalTok{].values[}\DecValTok{0}\NormalTok{]}
        \CommentTok{\#Discounted value of expected cost saving is added to cumulative expected revenue}
\NormalTok{        \_Revenue}\OperatorTok{=}\NormalTok{\_Revenue}\OperatorTok{+}\NormalTok{costsaving}\OperatorTok{/}\NormalTok{(np.power(}\DecValTok{1}\OperatorTok{+}\NormalTok{\_Discount,t))}
        \CommentTok{\#Expected interest cost}
\NormalTok{        cost}\OperatorTok{=}\NormalTok{proposedrate}\OperatorTok{*}\NormalTok{balanceremaining}
        \CommentTok{\#Remaining loan balance}
\NormalTok{        balanceremaining}\OperatorTok{=}\NormalTok{balanceremaining}\OperatorTok{{-}}\NormalTok{(payment}\OperatorTok{{-}}\NormalTok{cost)}
        \CommentTok{\#Discounted interest cost is added to cumulative expected cost of the loan}
\NormalTok{        \_LoanCost}\OperatorTok{=}\NormalTok{\_LoanCost}\OperatorTok{+}\NormalTok{cost}\OperatorTok{/}\NormalTok{(np.power(}\DecValTok{1}\OperatorTok{+}\NormalTok{\_Discount,t))}
    \CommentTok{\#Expected profit from PV ownership is given by cumulative discounted expected revenue minus installation cost (possibly reduced by subsidy) minus cumulative expected discounted cost of the loan}
\NormalTok{    profit}\OperatorTok{=}\NormalTok{\_Revenue}\OperatorTok{{-}}\NormalTok{((}\DecValTok{1}\OperatorTok{{-}}\NormalTok{\_Subsidy1)}\OperatorTok{*}\NormalTok{\_Params[}\StringTok{"PanelCost"}\NormalTok{].values[}\DecValTok{0}\NormalTok{])}\OperatorTok{{-}}\NormalTok{\_LoanCost}
    \CommentTok{\#For each agent, set variable deciles to the PV ownership rate in their own income decile}
\NormalTok{    deciles}\OperatorTok{=}\NormalTok{\_PVOwnershipDeciles[\_IncomeDeciles.astype(}\BuiltInTok{int}\NormalTok{)]}
    \CommentTok{\#Utility depends on profit, social influence component (deciles=share of other agents in decile who already have PV) and environmental attitude}
\NormalTok{    utility}\OperatorTok{=}\NormalTok{profit}\OperatorTok{+}\NormalTok{\_Influence}\OperatorTok{*}\NormalTok{(deciles}\OperatorTok{{-}}\NormalTok{(}\DecValTok{1}\OperatorTok{{-}}\NormalTok{deciles))}\OperatorTok{+}\NormalTok{\_Params[}\StringTok{"Beta"}\NormalTok{].values[}\DecValTok{0}\NormalTok{]}\OperatorTok{*}\NormalTok{\_Attitude}
    \CommentTok{\#Assume that agents who had already adopted PV before but whose panel has reached maximum age will adopt again if cumulative profit from the last panel was positive}
\NormalTok{    utility[(\_Adopted}\OperatorTok{==}\DecValTok{1}\NormalTok{) }\OperatorTok{\&}\NormalTok{ (\_CumulativeProfit}\OperatorTok{\textgreater{}}\DecValTok{0}\NormalTok{) }\OperatorTok{\&}\NormalTok{ (\_PV}\OperatorTok{==}\DecValTok{0}\NormalTok{)]}\OperatorTok{=}\DecValTok{1}
    \CommentTok{\#Re{-}set cumulative profit for agents who made a loss over the lifetime of their panel to 0}
\NormalTok{    \_CumulativeProfit[(\_Adopted}\OperatorTok{==}\DecValTok{1}\NormalTok{) }\OperatorTok{\&}\NormalTok{ (\_CumulativeProfit}\OperatorTok{\textless{}=}\DecValTok{0}\NormalTok{) }\OperatorTok{\&}\NormalTok{ (\_PV}\OperatorTok{==}\DecValTok{0}\NormalTok{)]}\OperatorTok{=}\DecValTok{0}
    \CommentTok{\#Agents who are credit rationed will not adopt}
\NormalTok{    utility[\_Rationed}\OperatorTok{==}\DecValTok{1}\NormalTok{]}\OperatorTok{=}\DecValTok{0}
    \CommentTok{\#Agents for whom PV is not technically feasible will not adopt}
\NormalTok{    utility[\_Feasible}\OperatorTok{==}\DecValTok{0}\NormalTok{]}\OperatorTok{=}\DecValTok{0}
    \CommentTok{\#Agents who already own PV will not adopt}
\NormalTok{    utility[\_PV}\OperatorTok{==}\DecValTok{1}\NormalTok{]}\OperatorTok{=}\DecValTok{0}
    \CommentTok{\#Find agents who will adopt}
\NormalTok{    adopters}\OperatorTok{=}\NormalTok{np.where(utility}\OperatorTok{\textgreater{}}\DecValTok{0}\NormalTok{)}
    \CommentTok{\#Find adopters who need a loan}
\NormalTok{    borrowers}\OperatorTok{=}\NormalTok{np.where((utility}\OperatorTok{\textgreater{}}\DecValTok{0}\NormalTok{) }\OperatorTok{\&}\NormalTok{ (loanneeded}\OperatorTok{\textgreater{}}\DecValTok{0}\NormalTok{))}
    \CommentTok{\#Set PV dummy for adopters to 1}
\NormalTok{    \_PV[adopters]}\OperatorTok{=}\DecValTok{1}
    \CommentTok{\#Set FIT for adopters (will be unchanged over the lifetime of the panel!)}
\NormalTok{    \_PanelFIT[adopters]}\OperatorTok{=}\NormalTok{\_FIT[}\BuiltInTok{str}\NormalTok{(\_year)].values[}\DecValTok{0}\NormalTok{]}
    \CommentTok{\#Indicates whether agent adopted PV at any point during the run; set to 1 for adopters}
\NormalTok{    \_Adopted[adopters]}\OperatorTok{=}\DecValTok{1}
    \CommentTok{\#Re{-}initialise cumulative profit for adopters}
\NormalTok{    \_CumulativeProfit[adopters]}\OperatorTok{={-}}\NormalTok{((}\DecValTok{1}\OperatorTok{{-}}\NormalTok{\_Subsidy1)}\OperatorTok{*}\NormalTok{\_Params[}\StringTok{"PanelCost"}\NormalTok{].values[}\DecValTok{0}\NormalTok{])}
    \CommentTok{\#Update liquidity for adopters, possibly including loan needed}
\NormalTok{    \_Liquidity[adopters]}\OperatorTok{=}\NormalTok{\_Liquidity[adopters]}\OperatorTok{{-}}\NormalTok{((}\DecValTok{1}\OperatorTok{{-}}\NormalTok{\_Subsidy1)}\OperatorTok{*}\NormalTok{\_Params[}\StringTok{"PanelCost"}\NormalTok{].values[}\DecValTok{0}\NormalTok{])}\OperatorTok{+}\NormalTok{loanneeded[adopters]}
    \CommentTok{\#Set loan rate for borrowers}
\NormalTok{    \_LoanRate[borrowers]}\OperatorTok{=}\NormalTok{proposedrate[borrowers]}\OperatorTok{*}\NormalTok{loanneeded[borrowers]}\OperatorTok{/}\NormalTok{(loanneeded[borrowers]}\OperatorTok{+}\NormalTok{\_Debt[borrowers])}\OperatorTok{+}\NormalTok{\_LoanRate[borrowers]}\OperatorTok{*}\NormalTok{\_Debt[borrowers]}\OperatorTok{/}\NormalTok{(loanneeded[borrowers]}\OperatorTok{+}\NormalTok{\_Debt[borrowers])}
    \CommentTok{\#Set debt level and debt service payments for borrowers}
\NormalTok{    \_Debt[borrowers]}\OperatorTok{=}\NormalTok{\_Debt[borrowers]}\OperatorTok{+}\NormalTok{loanneeded[borrowers]}
\NormalTok{    \_DebtService[borrowers]}\OperatorTok{=}\NormalTok{\_Debt[borrowers]}\OperatorTok{*}\NormalTok{(\_LoanRate[borrowers]}\OperatorTok{*}\NormalTok{(np.power(}\DecValTok{1}\OperatorTok{+}\NormalTok{\_LoanRate[borrowers],\_Params[}\StringTok{"PVMaxAge"}\NormalTok{].values[}\DecValTok{0}\NormalTok{])))}\OperatorTok{/}\NormalTok{((np.power(}\DecValTok{1}\OperatorTok{+}\NormalTok{\_LoanRate[borrowers],\_Params[}\StringTok{"PVMaxAge"}\NormalTok{].values[}\DecValTok{0}\NormalTok{]))}\OperatorTok{{-}}\DecValTok{1}\NormalTok{)}
    \ControlFlowTok{return}\NormalTok{[\_PV,\_Rationed,\_Liquidity,\_LoanRate,\_Debt,\_DebtService,\_Adopted,\_CumulativeProfit,\_PanelFIT]}
\end{Highlighting}
\end{Shaded}

\hypertarget{pv-deciles}{%
\section{PV Deciles}\label{pv-deciles}}

This function calculates the share of agents who own PV in each decile.
It is called at the end of a simulation period and, because an initial
value is needed for the adoption decision in the first simulation
period, once after function \texttt{IncomeGroups} during the first
simulation period. It determines the overall number of agents in each
income decile as well as the number of agents owning PV in each decile
in order to calculate a PV ownership rate for each decile.

\begin{Shaded}
\begin{Highlighting}[]
\KeywordTok{def}\NormalTok{ PVDeciles(\_Income\_p,\_PV,\_Positions\_d,\_Deciles,\_PVOwnershipDeciles,\_DecileMembers):}
    \CommentTok{\#Sort vector of income net of electricity cost in ascending order}
\NormalTok{    income\_psorted}\OperatorTok{=}\NormalTok{np.sort(\_Income\_p)}
    \CommentTok{\#Based on sorted income and number of agents in each decile, set cut{-}off points}
\NormalTok{    \_Deciles[}\DecValTok{0}\NormalTok{:}\DecValTok{9}\NormalTok{]}\OperatorTok{=}\NormalTok{income\_psorted[\_Positions\_d]}
\NormalTok{    \_Deciles[}\DecValTok{9}\NormalTok{]}\OperatorTok{=}\NormalTok{income\_psorted[((income\_psorted.size)}\OperatorTok{{-}}\DecValTok{1}\NormalTok{)]}
    \CommentTok{\#Iterate over deciles}
    \ControlFlowTok{for}\NormalTok{ i }\KeywordTok{in} \BuiltInTok{range}\NormalTok{(}\DecValTok{10}\NormalTok{):}
        \CommentTok{\#Determine which agents belong to decile i}
        \ControlFlowTok{if}\NormalTok{ i}\OperatorTok{==}\DecValTok{0}\NormalTok{:}
\NormalTok{            members}\OperatorTok{=}\NormalTok{np.where(\_Income\_p}\OperatorTok{\textless{}=}\NormalTok{\_Deciles[i])}
        \ControlFlowTok{else}\NormalTok{:}
\NormalTok{            members}\OperatorTok{=}\NormalTok{np.where((\_Income\_p}\OperatorTok{\textless{}=}\NormalTok{\_Deciles[i]) }\OperatorTok{\&}\NormalTok{ (\_Income\_p}\OperatorTok{\textgreater{}}\NormalTok{\_Deciles[(i}\OperatorTok{{-}}\DecValTok{1}\NormalTok{)]))}
        \CommentTok{\#Set the number of agents in decile i}
\NormalTok{        \_DecileMembers[i]}\OperatorTok{=}\NormalTok{members[}\DecValTok{0}\NormalTok{].size}
        \CommentTok{\#Set the number of agents owning PV in decile i}
\NormalTok{        \_PVOwnershipDeciles[i]}\OperatorTok{=}\NormalTok{\_PV[members[}\DecValTok{0}\NormalTok{]].}\BuiltInTok{sum}\NormalTok{()}
    \CommentTok{\#Calculate ownership rate by decile}
\NormalTok{    \_PVOwnershipDeciles}\OperatorTok{=}\NormalTok{\_PVOwnershipDeciles}\OperatorTok{/}\NormalTok{\_DecileMembers}
    \ControlFlowTok{return}\NormalTok{[\_PVOwnershipDeciles]}
\end{Highlighting}
\end{Shaded}

\hypertarget{calculate-statistics}{%
\section{Calculate statistics}\label{calculate-statistics}}

This function calculates a range of statistics, time-series of which
will be returned by the main function at the end of a simulation.

\begin{Shaded}
\begin{Highlighting}[]
\KeywordTok{def}\NormalTok{ CalculateStatistics(\_PV,\_Income,\_Debt,\_Liquidity,\_ExpectedPrice,\_Feasible,\_Rationed):}
    \CommentTok{\#Average income}
\NormalTok{    averageincome}\OperatorTok{=}\NormalTok{np.}\BuiltInTok{sum}\NormalTok{(\_Income)}\OperatorTok{/}\NormalTok{\_Income.size}
    \CommentTok{\#Average Liquidity}
\NormalTok{    averageliquidity}\OperatorTok{=}\NormalTok{np.}\BuiltInTok{sum}\NormalTok{(\_Liquidity)}\OperatorTok{/}\NormalTok{\_Liquidity.size}
    \CommentTok{\#Average expected price in t+1}
\NormalTok{    averageexpectedprice}\OperatorTok{=}\NormalTok{np.}\BuiltInTok{sum}\NormalTok{(\_ExpectedPrice)}\OperatorTok{/}\NormalTok{\_ExpectedPrice.size}
    \CommentTok{\#Average debt, only taking into account agents with positive debt}
    \ControlFlowTok{if}\NormalTok{ np.}\BuiltInTok{sum}\NormalTok{(\_Debt}\OperatorTok{\textgreater{}}\DecValTok{0}\NormalTok{)}\OperatorTok{\textgreater{}}\DecValTok{0}\NormalTok{:}
\NormalTok{        averagedebt}\OperatorTok{=}\NormalTok{np.}\BuiltInTok{sum}\NormalTok{(\_Debt[\_Debt}\OperatorTok{\textgreater{}}\DecValTok{0}\NormalTok{])}\OperatorTok{/}\NormalTok{np.}\BuiltInTok{sum}\NormalTok{(\_Debt}\OperatorTok{\textgreater{}}\DecValTok{0}\NormalTok{)}
    \ControlFlowTok{else}\NormalTok{:}
\NormalTok{        averagedebt}\OperatorTok{=}\DecValTok{0}
    \CommentTok{\#Adoption rate taking into account all agents}
\NormalTok{    adoptionrate1}\OperatorTok{=}\NormalTok{np.}\BuiltInTok{sum}\NormalTok{(\_PV)}\OperatorTok{/}\NormalTok{\_PV.size}
    \CommentTok{\#Adoption rate taking into account only agents for whom PV is technically feasible}
    \ControlFlowTok{if}\NormalTok{ np.}\BuiltInTok{sum}\NormalTok{(\_Feasible)}\OperatorTok{\textgreater{}}\DecValTok{0}\NormalTok{:}
\NormalTok{        adoptionrate2}\OperatorTok{=}\NormalTok{np.}\BuiltInTok{sum}\NormalTok{(\_PV)}\OperatorTok{/}\NormalTok{np.}\BuiltInTok{sum}\NormalTok{(\_Feasible)}
    \ControlFlowTok{else}\NormalTok{:}
\NormalTok{        adoptionrate2}\OperatorTok{=}\DecValTok{0}
    \CommentTok{\#Share of agents who are credit{-}rationed}
    \ControlFlowTok{if}\NormalTok{ np.}\BuiltInTok{sum}\NormalTok{((\_PV}\OperatorTok{==}\DecValTok{0}\NormalTok{) }\OperatorTok{\&}\NormalTok{ (\_Feasible}\OperatorTok{==}\DecValTok{1}\NormalTok{))}\OperatorTok{\textgreater{}}\DecValTok{0}\NormalTok{:}
\NormalTok{        rationing}\OperatorTok{=}\NormalTok{np.}\BuiltInTok{sum}\NormalTok{(\_Rationed)}\OperatorTok{/}\NormalTok{np.}\BuiltInTok{sum}\NormalTok{((\_PV}\OperatorTok{==}\DecValTok{0}\NormalTok{) }\OperatorTok{\&}\NormalTok{ (\_Feasible}\OperatorTok{==}\DecValTok{1}\NormalTok{))}
    \ControlFlowTok{else}\NormalTok{:}
\NormalTok{        rationing}\OperatorTok{=}\DecValTok{0}
    \ControlFlowTok{return}\NormalTok{(averageincome,averageliquidity,averageexpectedprice,averagedebt,adoptionrate1,adoptionrate2,rationing)}
\end{Highlighting}
\end{Shaded}

\hypertarget{main-model-function}{%
\section{Main model function}\label{main-model-function}}

This is the function which must be called to execute the model. It takes
several inputs, namely:

\begin{itemize}
\tightlist
\item
  \texttt{inputfile} - a csv file containing a a list of all model
  households and their initial characteristics
\item
  \texttt{propensitiesfile} - a csv file containing propensities to
  consume out of income by income percentile
\item
  \texttt{liquiditysharesfile} - a csv file containing the shares of
  saving going into liquid assets by income percentile
\item
  \texttt{fitfile} - a csv containing the value of the feed-in tariff
  for solar panels installed in each simulation year
\item
  \texttt{costfile} - a csv containing an annual time-series of
  installation cost of a solar panel
\item
  \texttt{ratefile} - a csv containing an annual time-series of central
  bank base interest rates
\item
  \texttt{pricefile} - a csv containing an annual time-series of
  electricity prices
\item
  \texttt{paramfile} - a csv file containing the names and values of all
  model parameters
\item
  \texttt{seed} - a seed for the pseudo-random number generator
\item
  \texttt{start} - the start year of the model simulation
\item
  \texttt{end} - the end year of the model simulation
\end{itemize}

The function begins by setting the seed for the pseudo-random number
generator, and then loads all necessary external input files. It then
initialises the model by generating vectors/arrays of all model
variables and setting initial values where necessary.

Subsequently, there is a single call to
\texttt{GenerateElectricityPrice} in order to generate a time-series of
electricity prices for the current model run. Subsequently, one of the
electricity price time-series is used to train the agents' expectations
formation algorithm. After this, the main simulation loop is entered. In
each simulation period, the model calls the functions described above
one after the other, updating the relevant model variables based on the
outputs of each function.

At the end of the simulation run, the time-series of aggregate
statistics to be returned are stacked into a single array and converted
to a dataframe, which is then saved in csv format, with the seed number
attached to the file name. Similarly, the time-series of PV adoption
rates by income decile are saved in csv format, with the seed number
attached to the file name.

\begin{Shaded}
\begin{Highlighting}[]
\KeywordTok{def}\NormalTok{ PVModel(inputfile}\OperatorTok{=}\StringTok{"inputs.csv"}\NormalTok{,propensitiesfile}\OperatorTok{=}\StringTok{"consumptionpropensities.csv"}\NormalTok{,liquiditysharesfile}\OperatorTok{=}\StringTok{"liquidityshares.csv"}\NormalTok{,fitfile}\OperatorTok{=}\StringTok{"FIT.csv"}\NormalTok{,costfile}\OperatorTok{=}\StringTok{"pvcost.csv"}\NormalTok{,ratefile}\OperatorTok{=}\StringTok{"baserate.csv"}\NormalTok{,pricefile}\OperatorTok{=}\StringTok{"electricityprice.csv"}\NormalTok{,paramfile}\OperatorTok{=}\StringTok{"parameters.csv"}\NormalTok{,seed}\OperatorTok{=}\DecValTok{1}\NormalTok{,runname}\OperatorTok{=}\StringTok{"test"}\NormalTok{,start}\OperatorTok{=}\DecValTok{2018}\NormalTok{,end}\OperatorTok{=}\DecValTok{2040}\NormalTok{):}
    \CommentTok{\#Set seed}
\NormalTok{    np.random.seed(seed)}
    
    \CommentTok{\#Load external input files}
\NormalTok{    Inputs}\OperatorTok{=}\NormalTok{pd.read\_csv(inputfile)}
\NormalTok{    Params}\OperatorTok{=}\NormalTok{pd.read\_csv(paramfile)}
\NormalTok{    Propensities}\OperatorTok{=}\NormalTok{np.ndarray.flatten(pd.read\_csv(propensitiesfile,header}\OperatorTok{=}\VariableTok{None}\NormalTok{).to\_numpy())}
\NormalTok{    LiquidityShares}\OperatorTok{=}\NormalTok{np.ndarray.flatten(pd.read\_csv(liquiditysharesfile,header}\OperatorTok{=}\VariableTok{None}\NormalTok{).to\_numpy())}
\NormalTok{    FIT}\OperatorTok{=}\NormalTok{pd.read\_csv(fitfile)}
\NormalTok{    PVCost}\OperatorTok{=}\NormalTok{pd.read\_csv(costfile)}
\NormalTok{    Baserate}\OperatorTok{=}\NormalTok{pd.read\_csv(ratefile)}
\NormalTok{    PriceEmp}\OperatorTok{=}\NormalTok{pd.read\_csv(pricefile)}

    \CommentTok{\#Initialise deterministic mean values for income}
\NormalTok{    Income\_f}\OperatorTok{=}\NormalTok{np.ndarray.flatten(Inputs[[}\StringTok{"income"}\NormalTok{]].to\_numpy())}
    \CommentTok{\#Create matrix for actual income}
\NormalTok{    Income}\OperatorTok{=}\NormalTok{np.zeros(shape}\OperatorTok{=}\NormalTok{(}\BuiltInTok{len}\NormalTok{(Inputs.index),(end}\OperatorTok{{-}}\NormalTok{start}\OperatorTok{+}\DecValTok{1}\NormalTok{)))}
    \CommentTok{\#Create matrix for income net of power cost}
\NormalTok{    Income\_p}\OperatorTok{=}\NormalTok{np.zeros(shape}\OperatorTok{=}\NormalTok{(}\BuiltInTok{len}\NormalTok{(Inputs.index),(end}\OperatorTok{{-}}\NormalTok{start}\OperatorTok{+}\DecValTok{1}\NormalTok{)))}
    \CommentTok{\#Set values for standard deviation of income}
\NormalTok{    Income\_sd}\OperatorTok{=}\NormalTok{np.ndarray.flatten(Inputs[[}\StringTok{"incomesd"}\NormalTok{]].to\_numpy())}
    \CommentTok{\#Initialise deterministic values of electricity consumption}
\NormalTok{    Powercost\_f}\OperatorTok{=}\NormalTok{np.ndarray.flatten(Inputs[[}\StringTok{"powercons"}\NormalTok{]].to\_numpy())}
    \CommentTok{\#Create matrix for actual power cost (including price)}
\NormalTok{    Powercost}\OperatorTok{=}\NormalTok{np.zeros(shape}\OperatorTok{=}\NormalTok{(}\BuiltInTok{len}\NormalTok{(Inputs.index),(end}\OperatorTok{{-}}\NormalTok{start}\OperatorTok{+}\DecValTok{1}\NormalTok{)))}
    \CommentTok{\#Create vector for current consumption}
\NormalTok{    Consumption}\OperatorTok{=}\NormalTok{np.zeros(}\BuiltInTok{len}\NormalTok{(Inputs.index))}
    \CommentTok{\#Create matrix for liquidity and set initial values}
\NormalTok{    Liquidity}\OperatorTok{=}\NormalTok{np.zeros(shape}\OperatorTok{=}\NormalTok{(}\BuiltInTok{len}\NormalTok{(Inputs.index),(end}\OperatorTok{{-}}\NormalTok{start}\OperatorTok{+}\DecValTok{1}\NormalTok{)))}
\NormalTok{    Liquidity[:,}\DecValTok{0}\NormalTok{]}\OperatorTok{=}\NormalTok{np.ndarray.flatten(Inputs[[}\StringTok{"liquidity"}\NormalTok{]].to\_numpy())}
    \CommentTok{\#Create vectors for debt, debt service payments and loan rates}
\NormalTok{    Debt}\OperatorTok{=}\NormalTok{np.zeros(}\BuiltInTok{len}\NormalTok{(Inputs.index))}
\NormalTok{    DebtService}\OperatorTok{=}\NormalTok{np.zeros(}\BuiltInTok{len}\NormalTok{(Inputs.index))}
\NormalTok{    LoanRate}\OperatorTok{=}\NormalTok{np.zeros(}\BuiltInTok{len}\NormalTok{(Inputs.index))}

    \CommentTok{\#Set trend rates at which electricity consumption and mean values of income will grow}
\NormalTok{    Trend}\OperatorTok{=}\NormalTok{np.ndarray.flatten(Inputs[[}\StringTok{"trend"}\NormalTok{]].to\_numpy())}
    \CommentTok{\#Set housing dummy (1=agent lives in house, 0=agent lives in flat)}
\NormalTok{    House}\OperatorTok{=}\NormalTok{np.ndarray.flatten(Inputs[[}\StringTok{"house"}\NormalTok{]].to\_numpy())}
    \CommentTok{\#Create matrix for current PV ownership dummies and set initial values (1=currently owns PV)}
\NormalTok{    PV}\OperatorTok{=}\NormalTok{np.zeros(shape}\OperatorTok{=}\NormalTok{(}\BuiltInTok{len}\NormalTok{(Inputs.index),(end}\OperatorTok{{-}}\NormalTok{start}\OperatorTok{+}\DecValTok{1}\NormalTok{)))}
\NormalTok{    PV[:,}\DecValTok{0}\NormalTok{]}\OperatorTok{=}\NormalTok{np.ndarray.flatten(Inputs[[}\StringTok{"pv"}\NormalTok{]].to\_numpy())}
    \CommentTok{\#Initialise age of solar panels}
\NormalTok{    PanelAge}\OperatorTok{=}\NormalTok{np.ndarray.flatten(Inputs[[}\StringTok{"panelage"}\NormalTok{]].to\_numpy())}
    \CommentTok{\#Set environmental attitudes}
\NormalTok{    Attitude}\OperatorTok{=}\NormalTok{np.ndarray.flatten(Inputs[[}\StringTok{"attitude"}\NormalTok{]].to\_numpy())}
    \CommentTok{\#Set strength of social influence for each agent}
\NormalTok{    Influence}\OperatorTok{=}\NormalTok{np.random.uniform(}\DecValTok{0}\NormalTok{,}\DecValTok{1}\NormalTok{,}\BuiltInTok{len}\NormalTok{(Inputs.index))}\OperatorTok{*}\NormalTok{(Params[}\StringTok{"MaxInfluence"}\NormalTok{].values[}\DecValTok{0}\NormalTok{]}\OperatorTok{{-}}\NormalTok{Params[}\StringTok{"MinInfluence"}\NormalTok{].values[}\DecValTok{0}\NormalTok{])}\OperatorTok{+}\NormalTok{Params[}\StringTok{"MinInfluence"}\NormalTok{].values[}\DecValTok{0}\NormalTok{]}
    \CommentTok{\#Set discount rate for each agent}
\NormalTok{    Discount}\OperatorTok{=}\NormalTok{np.ndarray.flatten(Inputs[[}\StringTok{"discount"}\NormalTok{]].to\_numpy())}
    \CommentTok{\#Create vector of dummies indicating whether PV is technically feasible}
\NormalTok{    Feasible}\OperatorTok{=}\NormalTok{np.zeros(}\BuiltInTok{len}\NormalTok{(Inputs.index))}
    \CommentTok{\#If agent already owns PV, it is feasible}
\NormalTok{    Feasible[PV[:,}\DecValTok{0}\NormalTok{]}\OperatorTok{==}\DecValTok{1}\NormalTok{]}\OperatorTok{=}\DecValTok{1}
    \CommentTok{\#If agent lives in house, it is feasible}
\NormalTok{    Feasible[House}\OperatorTok{==}\DecValTok{1}\NormalTok{]}\OperatorTok{=}\DecValTok{1}
    \CommentTok{\#If agent lives in flat, PV is feasible with a certain probability}
\NormalTok{    Feasible[(House}\OperatorTok{==}\DecValTok{0}\NormalTok{) }\OperatorTok{\&}\NormalTok{ (PV[:,}\DecValTok{0}\NormalTok{]}\OperatorTok{==}\DecValTok{0}\NormalTok{)]}\OperatorTok{=}\NormalTok{np.random.uniform(}\DecValTok{0}\NormalTok{,}\DecValTok{1}\NormalTok{,}\BuiltInTok{sum}\NormalTok{((House}\OperatorTok{==}\DecValTok{0}\NormalTok{) }\OperatorTok{\&}\NormalTok{ (PV[:,}\DecValTok{0}\NormalTok{]}\OperatorTok{==}\DecValTok{0}\NormalTok{)))}
\NormalTok{    Feasible[Feasible}\OperatorTok{\textgreater{}=}\NormalTok{(}\DecValTok{1}\OperatorTok{{-}}\NormalTok{Params[}\StringTok{"FeasibilityProb"}\NormalTok{].values[}\DecValTok{0}\NormalTok{])]}\OperatorTok{=}\DecValTok{1}
\NormalTok{    Feasible[Feasible}\OperatorTok{\textless{}}\NormalTok{(}\DecValTok{1}\OperatorTok{{-}}\NormalTok{Params[}\StringTok{"FeasibilityProb"}\NormalTok{].values[}\DecValTok{0}\NormalTok{])]}\OperatorTok{=}\DecValTok{0}
    \CommentTok{\#Create vector to hold feed{-}in tariffs earned by each agent}
\NormalTok{    PanelFIT}\OperatorTok{=}\NormalTok{np.zeros(}\BuiltInTok{len}\NormalTok{(Inputs.index))}
    \CommentTok{\#Find the oldest existing panel}
\NormalTok{    maxage}\OperatorTok{=}\BuiltInTok{max}\NormalTok{(PanelAge)}
    \ControlFlowTok{for}\NormalTok{ a }\KeywordTok{in} \BuiltInTok{range}\NormalTok{(maxage}\OperatorTok{+}\DecValTok{1}\NormalTok{):}
        \CommentTok{\#Set the adoption year}
\NormalTok{        yearadopted}\OperatorTok{=}\NormalTok{start}\OperatorTok{{-}}\NormalTok{(a}\OperatorTok{+}\DecValTok{1}\NormalTok{)}
        \CommentTok{\#If the adoption year is within the range of the time{-}series of FITs, set the FIT of all agents who adopted in that year to the appropriate value}
        \ControlFlowTok{if} \BuiltInTok{sum}\NormalTok{(}\BuiltInTok{str}\NormalTok{(yearadopted)}\OperatorTok{==}\NormalTok{FIT.columns)}\OperatorTok{==}\DecValTok{1}\NormalTok{:}
\NormalTok{            PanelFIT[(PV[:,}\DecValTok{0}\NormalTok{]}\OperatorTok{==}\DecValTok{1}\NormalTok{) }\OperatorTok{\&}\NormalTok{ (PanelAge}\OperatorTok{==}\NormalTok{a)]}\OperatorTok{=}\NormalTok{FIT[}\BuiltInTok{str}\NormalTok{(yearadopted)].values[}\DecValTok{0}\NormalTok{]}
        \CommentTok{\#Otherwise, set the FIT to the earliest available value}
        \ControlFlowTok{else}\NormalTok{:}
\NormalTok{            PanelFIT[(PV[:,}\DecValTok{0}\NormalTok{]}\OperatorTok{==}\DecValTok{1}\NormalTok{) }\OperatorTok{\&}\NormalTok{ (PanelAge}\OperatorTok{==}\NormalTok{a)]}\OperatorTok{=}\NormalTok{FIT.iloc[}\DecValTok{0}\NormalTok{,}\DecValTok{0}\NormalTok{]}

    \CommentTok{\#Set the cutoff points for income percentiles}
\NormalTok{    Positions\_p}\OperatorTok{=}\NormalTok{np.zeros(}\DecValTok{99}\NormalTok{)}
\NormalTok{    Positions\_p[:]}\OperatorTok{=}\NormalTok{np.array(}\BuiltInTok{range}\NormalTok{(}\DecValTok{1}\NormalTok{,}\DecValTok{100}\NormalTok{,}\DecValTok{1}\NormalTok{))}\OperatorTok{*}\NormalTok{(}\BuiltInTok{len}\NormalTok{(Inputs.index)}\OperatorTok{+}\DecValTok{1}\NormalTok{)}\OperatorTok{/}\DecValTok{100}\OperatorTok{{-}}\DecValTok{1}
\NormalTok{    Positions\_p}\OperatorTok{=}\NormalTok{np.}\BuiltInTok{round}\NormalTok{(Positions\_p)}
\NormalTok{    Positions\_p}\OperatorTok{=}\NormalTok{Positions\_p.astype(}\BuiltInTok{int}\NormalTok{)}
    \CommentTok{\#Create vector to hold income value associated with each percentile}
\NormalTok{    Percentiles}\OperatorTok{=}\NormalTok{np.zeros(}\DecValTok{100}\NormalTok{)}
    \CommentTok{\#Set the cutoff points for income deciles}
\NormalTok{    Positions\_d}\OperatorTok{=}\NormalTok{np.zeros(}\DecValTok{9}\NormalTok{)}
\NormalTok{    Positions\_d[:]}\OperatorTok{=}\NormalTok{np.array(}\BuiltInTok{range}\NormalTok{(}\DecValTok{1}\NormalTok{,}\DecValTok{10}\NormalTok{,}\DecValTok{1}\NormalTok{))}\OperatorTok{*}\NormalTok{(}\BuiltInTok{len}\NormalTok{(Inputs.index)}\OperatorTok{+}\DecValTok{1}\NormalTok{)}\OperatorTok{/}\DecValTok{10}\OperatorTok{{-}}\DecValTok{1}
\NormalTok{    Positions\_d}\OperatorTok{=}\NormalTok{np.}\BuiltInTok{round}\NormalTok{(Positions\_d)}
\NormalTok{    Positions\_d}\OperatorTok{=}\NormalTok{Positions\_d.astype(}\BuiltInTok{int}\NormalTok{)}
    \CommentTok{\#Create vector to hold income value associated with each decile}
\NormalTok{    Deciles}\OperatorTok{=}\NormalTok{np.zeros(}\DecValTok{10}\NormalTok{)}
    \CommentTok{\#Create vector to hold number of agents in each decile}
\NormalTok{    DecileMembers}\OperatorTok{=}\NormalTok{np.zeros(}\DecValTok{10}\NormalTok{)}
    \CommentTok{\#Create vector to hold PV ownership rate in each decile}
\NormalTok{    PVOwnershipDeciles}\OperatorTok{=}\NormalTok{np.zeros(}\DecValTok{10}\NormalTok{)}
    \CommentTok{\#Create vectors to hold the income deciles and percentiles of each agent}
\NormalTok{    IncomeDeciles}\OperatorTok{=}\NormalTok{np.zeros(}\BuiltInTok{len}\NormalTok{(Inputs.index))}
\NormalTok{    IncomePercentiles}\OperatorTok{=}\NormalTok{np.zeros(}\BuiltInTok{len}\NormalTok{(Inputs.index))}

    \CommentTok{\#Set the gain parameter used by each agent in the recursive least squares learning algorithm}
\NormalTok{    Gain}\OperatorTok{=}\NormalTok{np.random.uniform(}\DecValTok{0}\NormalTok{,}\DecValTok{1}\NormalTok{,}\BuiltInTok{len}\NormalTok{(Inputs.index))}\OperatorTok{*}\NormalTok{(Params[}\StringTok{"MaxGain"}\NormalTok{].values[}\DecValTok{0}\NormalTok{]}\OperatorTok{{-}}\NormalTok{Params[}\StringTok{"MinGain"}\NormalTok{].values[}\DecValTok{0}\NormalTok{])}\OperatorTok{+}\NormalTok{Params[}\StringTok{"MinGain"}\NormalTok{].values[}\DecValTok{0}\NormalTok{]}
    \CommentTok{\#Initialise agents\textquotesingle{} estimates of parameters and variance{-}covariance matrices}
\NormalTok{    ARPars}\OperatorTok{=}\NormalTok{np.array([[Params[}\StringTok{"ElectricityPriceTrend"}\NormalTok{].values[}\DecValTok{0}\NormalTok{]],[Params[}\StringTok{"ElectricityPriceAR"}\NormalTok{].values[}\DecValTok{0}\NormalTok{]]])}
\NormalTok{    ARPars}\OperatorTok{=}\NormalTok{[ARPars]}\OperatorTok{*}\BuiltInTok{len}\NormalTok{(Inputs.index)}
\NormalTok{    ARPars}\OperatorTok{=}\NormalTok{np.array(ARPars)}
\NormalTok{    ARMats}\OperatorTok{=}\NormalTok{np.array([[Params[}\StringTok{"ElectricityPriceMat11"}\NormalTok{].values[}\DecValTok{0}\NormalTok{],Params[}\StringTok{"ElectricityPriceMat12"}\NormalTok{].values[}\DecValTok{0}\NormalTok{]],[Params[}\StringTok{"ElectricityPriceMat21"}\NormalTok{].values[}\DecValTok{0}\NormalTok{],Params[}\StringTok{"ElectricityPriceMat22"}\NormalTok{].values[}\DecValTok{0}\NormalTok{]]])}
\NormalTok{    ARMats}\OperatorTok{=}\NormalTok{[ARMats]}\OperatorTok{*}\BuiltInTok{len}\NormalTok{(Inputs.index)}
\NormalTok{    ARMats}\OperatorTok{=}\NormalTok{np.array(ARMats)}
    \CommentTok{\#Create a matrix to hold agents\textquotesingle{} expected electricity prices}
\NormalTok{    ExpectedPrice}\OperatorTok{=}\NormalTok{np.zeros(shape}\OperatorTok{=}\NormalTok{(}\BuiltInTok{len}\NormalTok{(Inputs.index),Params[}\StringTok{"PVMaxAge"}\NormalTok{].values[}\DecValTok{0}\NormalTok{]))}

    \CommentTok{\#Initialise policy variables}
    \CommentTok{\#Subsidy on purchase price of panel}
\NormalTok{    Subsidy1}\OperatorTok{=}\DecValTok{0}
    \CommentTok{\#Additional subsidy on FIT}
\NormalTok{    Subsidy2}\OperatorTok{=}\DecValTok{0}
    \CommentTok{\#Change in maximum loan to value ratio}
\NormalTok{    CreditPolicy1}\OperatorTok{=}\DecValTok{0}
    \CommentTok{\#Change in maximum debt service to income ratio}
\NormalTok{    CreditPolicy2}\OperatorTok{=}\DecValTok{0}

    \CommentTok{\#Create vector to hold expected discounted cumulative revenue from owning PV}
\NormalTok{    Revenue}\OperatorTok{=}\NormalTok{np.zeros(}\BuiltInTok{len}\NormalTok{(Inputs.index))}
    \CommentTok{\#Create vector to hold expected discounted cumulative loan cost to purchase PV}
\NormalTok{    LoanCost}\OperatorTok{=}\NormalTok{np.zeros(}\BuiltInTok{len}\NormalTok{(Inputs.index))}
    \CommentTok{\#Create vector to hold dummies indicating whether agents are credit{-}rationed}
\NormalTok{    Rationed}\OperatorTok{=}\NormalTok{np.zeros(}\BuiltInTok{len}\NormalTok{(Inputs.index))}
    \CommentTok{\#Create vector to hold actual cumulative profit derived from owning PV}
\NormalTok{    CumulativeProfit}\OperatorTok{=}\NormalTok{np.zeros(}\BuiltInTok{len}\NormalTok{(Inputs.index))}
    \CommentTok{\#Create vector of dummies indicating whether agent has at any point owned PV}
\NormalTok{    Adopted}\OperatorTok{=}\NormalTok{np.zeros(}\BuiltInTok{len}\NormalTok{(Inputs.index))}
    \CommentTok{\#Set initial values for agents already owning PV at the start of the simulation}
\NormalTok{    Adopted[PV[:,}\DecValTok{0}\NormalTok{]}\OperatorTok{==}\DecValTok{1}\NormalTok{]}\OperatorTok{=}\DecValTok{1}
    \CommentTok{\#To initialise the cumulative profit on pre{-}existing solar panels, we divide agents owning PV into 4 groups}
    \CommentTok{\#Group 1 contains agents who own PV, whose electricity consumption is greater than the capacity of the panel, and whose FIT is lower than the starting value of the electricity price}
\NormalTok{    group1}\OperatorTok{=}\NormalTok{np.where((PV[:,}\DecValTok{0}\NormalTok{]}\OperatorTok{==}\DecValTok{1}\NormalTok{) }\OperatorTok{\&}\NormalTok{ (Powercost\_f}\OperatorTok{\textgreater{}=}\NormalTok{Params[}\StringTok{"PVCapacity"}\NormalTok{].values[}\DecValTok{0}\NormalTok{]) }\OperatorTok{\&}\NormalTok{ ((PanelFIT}\OperatorTok{+}\NormalTok{Subsidy2)}\OperatorTok{\textless{}}\NormalTok{PriceEmp[}\BuiltInTok{str}\NormalTok{(start}\OperatorTok{{-}}\DecValTok{1}\NormalTok{)].values[}\DecValTok{0}\NormalTok{]))}
    \CommentTok{\#Group 2 contains agents who own PV, whose electricity consumption is greater than the capacity of the panel, and whose FIT is higher than the starting value of the electricity price}
\NormalTok{    group2}\OperatorTok{=}\NormalTok{np.where((PV[:,}\DecValTok{0}\NormalTok{]}\OperatorTok{==}\DecValTok{1}\NormalTok{) }\OperatorTok{\&}\NormalTok{ (Powercost\_f}\OperatorTok{\textgreater{}=}\NormalTok{Params[}\StringTok{"PVCapacity"}\NormalTok{].values[}\DecValTok{0}\NormalTok{]) }\OperatorTok{\&}\NormalTok{ ((PanelFIT}\OperatorTok{+}\NormalTok{Subsidy2)}\OperatorTok{\textgreater{}=}\NormalTok{PriceEmp[}\BuiltInTok{str}\NormalTok{(start}\OperatorTok{{-}}\DecValTok{1}\NormalTok{)].values[}\DecValTok{0}\NormalTok{]))}
    \CommentTok{\#Group 3 contains agents who own PV, whose electricity consumption is smaller than the capacity of the panel, and whose FIT is lower than the starting value of the electricity price}
\NormalTok{    group3}\OperatorTok{=}\NormalTok{np.where((PV[:,}\DecValTok{0}\NormalTok{]}\OperatorTok{==}\DecValTok{1}\NormalTok{) }\OperatorTok{\&}\NormalTok{ (Powercost\_f}\OperatorTok{\textless{}}\NormalTok{Params[}\StringTok{"PVCapacity"}\NormalTok{].values[}\DecValTok{0}\NormalTok{]) }\OperatorTok{\&}\NormalTok{ ((PanelFIT}\OperatorTok{+}\NormalTok{Subsidy2)}\OperatorTok{\textless{}}\NormalTok{PriceEmp[}\BuiltInTok{str}\NormalTok{(start}\OperatorTok{{-}}\DecValTok{1}\NormalTok{)].values[}\DecValTok{0}\NormalTok{]))}
    \CommentTok{\#Group 4 contains agents who own PV, whose electricity consumption is lower than the capacity of the panel, and whose FIT is higher than the starting value of the electricity price}
\NormalTok{    group4}\OperatorTok{=}\NormalTok{np.where((PV[:,}\DecValTok{0}\NormalTok{]}\OperatorTok{==}\DecValTok{1}\NormalTok{) }\OperatorTok{\&}\NormalTok{ (Powercost\_f}\OperatorTok{\textless{}}\NormalTok{Params[}\StringTok{"PVCapacity"}\NormalTok{].values[}\DecValTok{0}\NormalTok{]) }\OperatorTok{\&}\NormalTok{ ((PanelFIT}\OperatorTok{+}\NormalTok{Subsidy2)}\OperatorTok{\textgreater{}=}\NormalTok{PriceEmp[}\BuiltInTok{str}\NormalTok{(start}\OperatorTok{{-}}\DecValTok{1}\NormalTok{)].values[}\DecValTok{0}\NormalTok{]))}
    \CommentTok{\#Cumulative profit of agents in group 1 is initialised as share of own consumption times capacity of the panel times initial electricity price plus remaining capacity times FIT}
\NormalTok{    CumulativeProfit[group1]}\OperatorTok{=}\NormalTok{Params[}\StringTok{"PVCons"}\NormalTok{].values[}\DecValTok{0}\NormalTok{]}\OperatorTok{*}\NormalTok{Params[}\StringTok{"PVCapacity"}\NormalTok{].values[}\DecValTok{0}\NormalTok{]}\OperatorTok{*}\NormalTok{PriceEmp[}\BuiltInTok{str}\NormalTok{(start}\OperatorTok{{-}}\DecValTok{1}\NormalTok{)].values[}\DecValTok{0}\NormalTok{]}\OperatorTok{+}\NormalTok{(}\DecValTok{1}\OperatorTok{{-}}\NormalTok{Params[}\StringTok{"PVCons"}\NormalTok{].values[}\DecValTok{0}\NormalTok{])}\OperatorTok{*}\NormalTok{Params[}\StringTok{"PVCapacity"}\NormalTok{].values[}\DecValTok{0}\NormalTok{]}\OperatorTok{*}\NormalTok{(PanelFIT[group1]}\OperatorTok{+}\NormalTok{Subsidy2)}
    \CommentTok{\#Cumulative profit of agents in group 2 is initialised as capacity of panel times FIT}
\NormalTok{    CumulativeProfit[group2]}\OperatorTok{=}\NormalTok{Params[}\StringTok{"PVCapacity"}\NormalTok{].values[}\DecValTok{0}\NormalTok{]}\OperatorTok{*}\NormalTok{(PanelFIT[group2]}\OperatorTok{+}\NormalTok{Subsidy2)}
    \CommentTok{\#Cumulative profit of agents in group 3 is initialised as minimum between share of own consumption times capacity of the panel and electricity consumption times initial electricity price plus remaining capacity times FIT}
\NormalTok{    CumulativeProfit[group3]}\OperatorTok{=}\NormalTok{np.minimum(Params[}\StringTok{"PVCons"}\NormalTok{].values[}\DecValTok{0}\NormalTok{]}\OperatorTok{*}\NormalTok{Params[}\StringTok{"PVCapacity"}\NormalTok{].values[}\DecValTok{0}\NormalTok{],Powercost\_f[group3])}\OperatorTok{*}\NormalTok{PriceEmp[}\BuiltInTok{str}\NormalTok{(start}\OperatorTok{{-}}\DecValTok{1}\NormalTok{)].values[}\DecValTok{0}\NormalTok{]}\OperatorTok{+}\NormalTok{(PanelFIT[group3]}\OperatorTok{+}\NormalTok{Subsidy2)}\OperatorTok{*}\NormalTok{(Params[}\StringTok{"PVCapacity"}\NormalTok{].values[}\DecValTok{0}\NormalTok{]}\OperatorTok{{-}}\NormalTok{np.minimum(Params[}\StringTok{"PVCons"}\NormalTok{].values[}\DecValTok{0}\NormalTok{]}\OperatorTok{*}\NormalTok{Params[}\StringTok{"PVCapacity"}\NormalTok{].values[}\DecValTok{0}\NormalTok{],Powercost\_f[group3]))}
    \CommentTok{\#Cumulative profit of agents in group 4 is initialised as capacity of panel times FIT}
\NormalTok{    CumulativeProfit[group4]}\OperatorTok{=}\NormalTok{Params[}\StringTok{"PVCapacity"}\NormalTok{].values[}\DecValTok{0}\NormalTok{]}\OperatorTok{*}\NormalTok{(PanelFIT[group4]}\OperatorTok{+}\NormalTok{Subsidy2)}
    \CommentTok{\#For simplicity, assume that agents earned this profit in all previous periods}
\NormalTok{    CumulativeProfit}\OperatorTok{=}\NormalTok{CumulativeProfit}\OperatorTok{*}\NormalTok{PanelAge}
    \CommentTok{\#Subtract from cumulative profit the cost of PV from the year in which it was installed}
    \ControlFlowTok{for}\NormalTok{ a }\KeywordTok{in} \BuiltInTok{range}\NormalTok{(maxage}\OperatorTok{+}\DecValTok{1}\NormalTok{):}
\NormalTok{        yearadopted}\OperatorTok{=}\NormalTok{start}\OperatorTok{{-}}\NormalTok{(a}\OperatorTok{+}\DecValTok{1}\NormalTok{)}
        \ControlFlowTok{if} \BuiltInTok{sum}\NormalTok{(}\BuiltInTok{str}\NormalTok{(yearadopted)}\OperatorTok{==}\NormalTok{PVCost.columns)}\OperatorTok{==}\DecValTok{1}\NormalTok{:}
\NormalTok{            CumulativeProfit[(PV[:,}\DecValTok{0}\NormalTok{]}\OperatorTok{==}\DecValTok{1}\NormalTok{) }\OperatorTok{\&}\NormalTok{ (PanelAge}\OperatorTok{==}\NormalTok{a)]}\OperatorTok{=}\NormalTok{CumulativeProfit[(PV[:,}\DecValTok{0}\NormalTok{]}\OperatorTok{==}\DecValTok{1}\NormalTok{) }\OperatorTok{\&}\NormalTok{ (PanelAge}\OperatorTok{==}\NormalTok{a)]}\OperatorTok{{-}}\NormalTok{(}\DecValTok{1}\OperatorTok{{-}}\NormalTok{Subsidy1)}\OperatorTok{*}\NormalTok{PVCost[}\BuiltInTok{str}\NormalTok{(yearadopted)].values[}\DecValTok{0}\NormalTok{]}
        \CommentTok{\#If a panel is older than the earliest available value for panel cost, use the earliest value}
        \ControlFlowTok{else}\NormalTok{:}
\NormalTok{            CumulativeProfit[(PV[:,}\DecValTok{0}\NormalTok{]}\OperatorTok{==}\DecValTok{1}\NormalTok{) }\OperatorTok{\&}\NormalTok{ (PanelAge}\OperatorTok{==}\NormalTok{a)]}\OperatorTok{=}\NormalTok{CumulativeProfit[(PV[:,}\DecValTok{0}\NormalTok{]}\OperatorTok{==}\DecValTok{1}\NormalTok{) }\OperatorTok{\&}\NormalTok{ (PanelAge}\OperatorTok{==}\NormalTok{a)]}\OperatorTok{{-}}\NormalTok{(}\DecValTok{1}\OperatorTok{{-}}\NormalTok{Subsidy1)}\OperatorTok{*}\NormalTok{PVCost.iloc[}\DecValTok{0}\NormalTok{,}\DecValTok{0}\NormalTok{]}


    \CommentTok{\#Initialise aggregate statistics}
\NormalTok{    AverageIncome}\OperatorTok{=}\NormalTok{np.zeros((end}\OperatorTok{{-}}\NormalTok{start}\OperatorTok{+}\DecValTok{1}\NormalTok{))}
\NormalTok{    AverageLiquidity}\OperatorTok{=}\NormalTok{np.zeros((end}\OperatorTok{{-}}\NormalTok{start}\OperatorTok{+}\DecValTok{1}\NormalTok{))}
\NormalTok{    AverageExpectedPrice}\OperatorTok{=}\NormalTok{np.zeros((end}\OperatorTok{{-}}\NormalTok{start}\OperatorTok{+}\DecValTok{1}\NormalTok{))}
\NormalTok{    AverageDebt}\OperatorTok{=}\NormalTok{np.zeros((end}\OperatorTok{{-}}\NormalTok{start}\OperatorTok{+}\DecValTok{1}\NormalTok{))}
\NormalTok{    AdoptionRate1}\OperatorTok{=}\NormalTok{np.zeros((end}\OperatorTok{{-}}\NormalTok{start}\OperatorTok{+}\DecValTok{1}\NormalTok{))}
\NormalTok{    AdoptionRate2}\OperatorTok{=}\NormalTok{np.zeros((end}\OperatorTok{{-}}\NormalTok{start}\OperatorTok{+}\DecValTok{1}\NormalTok{))}
\NormalTok{    AdoptionRateDeciles}\OperatorTok{=}\NormalTok{np.zeros(((end}\OperatorTok{{-}}\NormalTok{start}\OperatorTok{+}\DecValTok{1}\NormalTok{),}\DecValTok{10}\NormalTok{))}
\NormalTok{    ChangeAdoptionRate1}\OperatorTok{=}\NormalTok{np.zeros((end}\OperatorTok{{-}}\NormalTok{start}\OperatorTok{+}\DecValTok{1}\NormalTok{))}
\NormalTok{    ChangeAdoptionRate2}\OperatorTok{=}\NormalTok{np.zeros((end}\OperatorTok{{-}}\NormalTok{start}\OperatorTok{+}\DecValTok{1}\NormalTok{))}
\NormalTok{    RationedShare}\OperatorTok{=}\NormalTok{np.zeros((end}\OperatorTok{{-}}\NormalTok{start}\OperatorTok{+}\DecValTok{1}\NormalTok{))}

    \CommentTok{\#Generate two electricity price time{-}series}
\NormalTok{    PriceReturn}\OperatorTok{=}\NormalTok{GenerateElectricityPrice(\_PriceEmp}\OperatorTok{=}\NormalTok{PriceEmp.copy(),\_Params}\OperatorTok{=}\NormalTok{Params.copy(),\_length}\OperatorTok{=}\NormalTok{(end}\OperatorTok{{-}}\NormalTok{start}\OperatorTok{+}\DecValTok{3}\NormalTok{),\_start}\OperatorTok{=}\NormalTok{start)}
    \CommentTok{\#Price series to be used in the simulation}
\NormalTok{    Price}\OperatorTok{=}\NormalTok{PriceReturn[}\DecValTok{0}\NormalTok{]}
    \CommentTok{\#Price series used to train agents\textquotesingle{} expectations formation}
\NormalTok{    PriceTrain}\OperatorTok{=}\NormalTok{PriceReturn[}\DecValTok{1}\NormalTok{]}
    \CommentTok{\#Use PriceTrain to train agents\textquotesingle{} expectations formation using a number of time{-}steps equal to the length of the actual simulation}
    \ControlFlowTok{for}\NormalTok{ period }\KeywordTok{in} \BuiltInTok{range}\NormalTok{((end}\OperatorTok{{-}}\NormalTok{start}\OperatorTok{+}\DecValTok{1}\NormalTok{)):}
\NormalTok{        ExpectationReturn}\OperatorTok{=}\NormalTok{FormExpectation(\_ARPars}\OperatorTok{=}\NormalTok{ARPars.copy(),\_ARMats}\OperatorTok{=}\NormalTok{ARMats.copy(),\_Price}\OperatorTok{=}\NormalTok{PriceTrain.copy(),\_Gain}\OperatorTok{=}\NormalTok{Gain.copy(),\_ExpectedPrice}\OperatorTok{=}\NormalTok{ExpectedPrice.copy(),\_period}\OperatorTok{=}\NormalTok{period)}
\NormalTok{        ARMats}\OperatorTok{=}\NormalTok{ExpectationReturn[}\DecValTok{0}\NormalTok{]}
\NormalTok{        ARPars}\OperatorTok{=}\NormalTok{ExpectationReturn[}\DecValTok{1}\NormalTok{]}
\NormalTok{        ExpectedPrice}\OperatorTok{=}\NormalTok{ExpectationReturn[}\DecValTok{2}\NormalTok{]}
    
    \CommentTok{\#Enter main model loop}
    \CommentTok{\#Set the current year to the start year supplied to the function}
\NormalTok{    year}\OperatorTok{=}\NormalTok{start}
    \ControlFlowTok{for}\NormalTok{ period }\KeywordTok{in} \BuiltInTok{range}\NormalTok{((end}\OperatorTok{{-}}\NormalTok{start}\OperatorTok{+}\DecValTok{1}\NormalTok{)):}
        \CommentTok{\#Set the purchase and installation cost of a solar panel equal to the value for the current year}
\NormalTok{        Params[}\StringTok{"PanelCost"}\NormalTok{]}\OperatorTok{=}\NormalTok{PVCost[}\BuiltInTok{str}\NormalTok{(year)].values[}\DecValTok{0}\NormalTok{]}
        \CommentTok{\#Set starting value for liquid wealth and PV equal to closing value from last period}
        \ControlFlowTok{if}\NormalTok{ period}\OperatorTok{\textgreater{}}\DecValTok{0}\NormalTok{:}
\NormalTok{            Liquidity[:,period]}\OperatorTok{=}\NormalTok{Liquidity[:,(period}\OperatorTok{{-}}\DecValTok{1}\NormalTok{)]}
\NormalTok{            PV[:,period]}\OperatorTok{=}\NormalTok{PV[:,(period}\OperatorTok{{-}}\DecValTok{1}\NormalTok{)]}
        \CommentTok{\#Set policy variables}
\NormalTok{        PolicyReturn}\OperatorTok{=}\NormalTok{SetPolicy(\_Subsidy1}\OperatorTok{=}\NormalTok{Subsidy1,\_Subsidy2}\OperatorTok{=}\NormalTok{Subsidy2,\_CreditPolicy1}\OperatorTok{=}\NormalTok{CreditPolicy1,\_CreditPolicy2}\OperatorTok{=}\NormalTok{CreditPolicy2,\_year}\OperatorTok{=}\NormalTok{year,\_Params}\OperatorTok{=}\NormalTok{Params.copy())}
\NormalTok{        Subsidy1}\OperatorTok{=}\NormalTok{PolicyReturn[}\DecValTok{0}\NormalTok{]}
\NormalTok{        Subsidy2}\OperatorTok{=}\NormalTok{PolicyReturn[}\DecValTok{1}\NormalTok{]}
\NormalTok{        CreditPolicy1}\OperatorTok{=}\NormalTok{PolicyReturn[}\DecValTok{2}\NormalTok{]}
\NormalTok{        CreditPolicy2}\OperatorTok{=}\NormalTok{PolicyReturn[}\DecValTok{3}\NormalTok{]}
        \CommentTok{\#Calculate income and electricity cost}
\NormalTok{        IncomePowercostReturn}\OperatorTok{=}\NormalTok{IncomePowercost(\_Income\_f}\OperatorTok{=}\NormalTok{Income\_f.copy(),\_Trend}\OperatorTok{=}\NormalTok{Trend.copy(),\_Powercost\_f}\OperatorTok{=}\NormalTok{Powercost\_f.copy(),\_Income\_sd}\OperatorTok{=}\NormalTok{Income\_sd.copy(),\_CumulativeProfit}\OperatorTok{=}\NormalTok{CumulativeProfit.copy(),\_PV}\OperatorTok{=}\NormalTok{PV[:,period].copy(),\_Price}\OperatorTok{=}\NormalTok{Price[(period}\OperatorTok{+}\DecValTok{2}\NormalTok{)].copy(),\_PanelFIT}\OperatorTok{=}\NormalTok{PanelFIT.copy(),\_Subsidy2}\OperatorTok{=}\NormalTok{Subsidy2,\_Params}\OperatorTok{=}\NormalTok{Params.copy())}
\NormalTok{        Income[:,period]}\OperatorTok{=}\NormalTok{IncomePowercostReturn[}\DecValTok{0}\NormalTok{]}
\NormalTok{        Powercost[:,period]}\OperatorTok{=}\NormalTok{IncomePowercostReturn[}\DecValTok{1}\NormalTok{]}
\NormalTok{        Income\_p[:,period]}\OperatorTok{=}\NormalTok{IncomePowercostReturn[}\DecValTok{2}\NormalTok{]}
\NormalTok{        CumulativeProfit[:]}\OperatorTok{=}\NormalTok{IncomePowercostReturn[}\DecValTok{3}\NormalTok{]}
\NormalTok{        Income\_f[:]}\OperatorTok{=}\NormalTok{IncomePowercostReturn[}\DecValTok{4}\NormalTok{]}
\NormalTok{        Powercost\_f[:]}\OperatorTok{=}\NormalTok{IncomePowercostReturn[}\DecValTok{5}\NormalTok{]}
        \CommentTok{\#Calculate income distribution statistics}
\NormalTok{        IncomeGroupReturn}\OperatorTok{=}\NormalTok{IncomeGroups(\_Income\_p}\OperatorTok{=}\NormalTok{Income\_p[:,period].copy(),\_Positions\_p}\OperatorTok{=}\NormalTok{Positions\_p.copy(),\_Percentiles}\OperatorTok{=}\NormalTok{Percentiles.copy(),\_Positions\_d}\OperatorTok{=}\NormalTok{Positions\_d.copy(),\_Deciles}\OperatorTok{=}\NormalTok{Deciles.copy(),\_IncomePercentiles}\OperatorTok{=}\NormalTok{IncomePercentiles.copy(),\_IncomeDeciles}\OperatorTok{=}\NormalTok{IncomeDeciles.copy())}
\NormalTok{        IncomePercentiles[:]}\OperatorTok{=}\NormalTok{IncomeGroupReturn[}\DecValTok{0}\NormalTok{]}
\NormalTok{        IncomeDeciles[:]}\OperatorTok{=}\NormalTok{IncomeGroupReturn[}\DecValTok{1}\NormalTok{]}
        \CommentTok{\#In first simulation period, initialise PV ownership by decile}
        \ControlFlowTok{if}\NormalTok{ period}\OperatorTok{==}\DecValTok{0}\NormalTok{:}
\NormalTok{            PVDecileReturn}\OperatorTok{=}\NormalTok{PVDeciles(\_Income\_p}\OperatorTok{=}\NormalTok{Income\_p[:,period].copy(),\_PV}\OperatorTok{=}\NormalTok{PV[:,period].copy(),\_Positions\_d}\OperatorTok{=}\NormalTok{Positions\_d.copy(),\_PVOwnershipDeciles}\OperatorTok{=}\NormalTok{PVOwnershipDeciles.copy(),\_Deciles}\OperatorTok{=}\NormalTok{Deciles.copy(),\_DecileMembers}\OperatorTok{=}\NormalTok{DecileMembers.copy())}
\NormalTok{            PVOwnershipDeciles[:]}\OperatorTok{=}\NormalTok{PVDecileReturn[}\DecValTok{0}\NormalTok{]}
        \CommentTok{\#Calculate consumption and saving}
\NormalTok{        ConsumptionSavingReturn}\OperatorTok{=}\NormalTok{ConsumptionSaving(\_Income\_p}\OperatorTok{=}\NormalTok{Income\_p[:,period].copy(),\_Propensities}\OperatorTok{=}\NormalTok{Propensities.copy(),\_IncomePercentiles}\OperatorTok{=}\NormalTok{IncomePercentiles.copy(),\_Consumption}\OperatorTok{=}\NormalTok{Consumption.copy(),\_Income}\OperatorTok{=}\NormalTok{Income[:,period].copy(),\_Powercost}\OperatorTok{=}\NormalTok{Powercost[:,period].copy(),\_Liquidity}\OperatorTok{=}\NormalTok{Liquidity[:,period].copy(),\_Debt}\OperatorTok{=}\NormalTok{Debt.copy(),\_DebtService}\OperatorTok{=}\NormalTok{DebtService.copy(),\_LoanRate}\OperatorTok{=}\NormalTok{LoanRate.copy(),\_PanelAge}\OperatorTok{=}\NormalTok{PanelAge.copy(),\_CumulativeProfit}\OperatorTok{=}\NormalTok{CumulativeProfit.copy(),\_LiquidityShares}\OperatorTok{=}\NormalTok{LiquidityShares.copy(),\_Params}\OperatorTok{=}\NormalTok{Params.copy(),\_period}\OperatorTok{=}\NormalTok{period)}
\NormalTok{        Liquidity[:,period]}\OperatorTok{=}\NormalTok{ConsumptionSavingReturn[}\DecValTok{0}\NormalTok{]}
\NormalTok{        Debt[:]}\OperatorTok{=}\NormalTok{ConsumptionSavingReturn[}\DecValTok{1}\NormalTok{]}
\NormalTok{        DebtService[:]}\OperatorTok{=}\NormalTok{ConsumptionSavingReturn[}\DecValTok{2}\NormalTok{]}
\NormalTok{        LoanRate[:]}\OperatorTok{=}\NormalTok{ConsumptionSavingReturn[}\DecValTok{3}\NormalTok{]}
\NormalTok{        Consumption[:]}\OperatorTok{=}\NormalTok{ConsumptionSavingReturn[}\DecValTok{4}\NormalTok{]}
\NormalTok{        CumulativeProfit[:]}\OperatorTok{=}\NormalTok{ConsumptionSavingReturn[}\DecValTok{5}\NormalTok{]}
        \CommentTok{\#Increment age of existing panels}
\NormalTok{        PanelAgeReturn}\OperatorTok{=}\NormalTok{AugmentPanelAge(\_PV}\OperatorTok{=}\NormalTok{PV[:,period].copy(),\_PanelAge}\OperatorTok{=}\NormalTok{PanelAge.copy(),\_PanelFIT}\OperatorTok{=}\NormalTok{PanelFIT.copy(),\_Params}\OperatorTok{=}\NormalTok{Params.copy())}
\NormalTok{        PanelAge}\OperatorTok{=}\NormalTok{PanelAgeReturn[}\DecValTok{0}\NormalTok{]}
\NormalTok{        PV[:,period]}\OperatorTok{=}\NormalTok{PanelAgeReturn[}\DecValTok{1}\NormalTok{]}
\NormalTok{        PanelFIT}\OperatorTok{=}\NormalTok{PanelAgeReturn[}\DecValTok{2}\NormalTok{]}
        \CommentTok{\#Form expectations}
\NormalTok{        ExpectationReturn}\OperatorTok{=}\NormalTok{FormExpectation(\_ARPars}\OperatorTok{=}\NormalTok{ARPars.copy(),\_ARMats}\OperatorTok{=}\NormalTok{ARMats.copy(),\_Price}\OperatorTok{=}\NormalTok{Price.copy(),\_Gain}\OperatorTok{=}\NormalTok{Gain.copy(),\_ExpectedPrice}\OperatorTok{=}\NormalTok{ExpectedPrice.copy(),\_period}\OperatorTok{=}\NormalTok{period)}
\NormalTok{        ARMats}\OperatorTok{=}\NormalTok{ExpectationReturn[}\DecValTok{0}\NormalTok{]}
\NormalTok{        ARPars}\OperatorTok{=}\NormalTok{ExpectationReturn[}\DecValTok{1}\NormalTok{]}
\NormalTok{        ExpectedPrice}\OperatorTok{=}\NormalTok{ExpectationReturn[}\DecValTok{2}\NormalTok{]}
        \CommentTok{\#Adoption decision}
\NormalTok{        AdoptionReturn}\OperatorTok{=}\NormalTok{AdoptionDecision(\_PV}\OperatorTok{=}\NormalTok{PV[:,period].copy(),\_Feasible}\OperatorTok{=}\NormalTok{Feasible.copy(),\_Liquidity}\OperatorTok{=}\NormalTok{Liquidity[:,period].copy(),\_Rationed}\OperatorTok{=}\NormalTok{Rationed.copy(),\_Debt}\OperatorTok{=}\NormalTok{Debt.copy(),\_Income}\OperatorTok{=}\NormalTok{Income[:,period].copy(),\_Income\_f}\OperatorTok{=}\NormalTok{Income\_f.copy(),\_Powercost\_f}\OperatorTok{=}\NormalTok{Powercost\_f.copy(),\_Trend}\OperatorTok{=}\NormalTok{Trend.copy(),\_ExpectedPrice}\OperatorTok{=}\NormalTok{ExpectedPrice.copy(),\_Discount}\OperatorTok{=}\NormalTok{Discount.copy(),\_Revenue}\OperatorTok{=}\NormalTok{Revenue.copy(),\_LoanCost}\OperatorTok{=}\NormalTok{LoanCost.copy(),\_PVOwnershipDeciles}\OperatorTok{=}\NormalTok{PVOwnershipDeciles.copy(),\_IncomeDeciles}\OperatorTok{=}\NormalTok{IncomeDeciles.copy(),\_LoanRate}\OperatorTok{=}\NormalTok{LoanRate.copy(),\_DebtService}\OperatorTok{=}\NormalTok{DebtService.copy(),\_Influence}\OperatorTok{=}\NormalTok{Influence.copy(),\_Attitude}\OperatorTok{=}\NormalTok{Attitude.copy(),\_Adopted}\OperatorTok{=}\NormalTok{Adopted.copy(),\_CumulativeProfit}\OperatorTok{=}\NormalTok{CumulativeProfit.copy(),\_FIT}\OperatorTok{=}\NormalTok{FIT.copy(),\_year}\OperatorTok{=}\NormalTok{year,\_PanelFIT}\OperatorTok{=}\NormalTok{PanelFIT.copy(),\_Baserate}\OperatorTok{=}\NormalTok{Baserate.copy(),\_Subsidy1}\OperatorTok{=}\NormalTok{Subsidy1,\_Subsidy2}\OperatorTok{=}\NormalTok{Subsidy2,\_CreditPolicy1}\OperatorTok{=}\NormalTok{CreditPolicy1,\_CreditPolicy2}\OperatorTok{=}\NormalTok{CreditPolicy2,\_Params}\OperatorTok{=}\NormalTok{Params.copy())}
\NormalTok{        PV[:,period]}\OperatorTok{=}\NormalTok{AdoptionReturn[}\DecValTok{0}\NormalTok{]}
\NormalTok{        Rationed[:]}\OperatorTok{=}\NormalTok{AdoptionReturn[}\DecValTok{1}\NormalTok{]}
\NormalTok{        Liquidity[:,period]}\OperatorTok{=}\NormalTok{AdoptionReturn[}\DecValTok{2}\NormalTok{]}
\NormalTok{        LoanRate[:]}\OperatorTok{=}\NormalTok{AdoptionReturn[}\DecValTok{3}\NormalTok{]}
\NormalTok{        Debt[:]}\OperatorTok{=}\NormalTok{AdoptionReturn[}\DecValTok{4}\NormalTok{]}
\NormalTok{        DebtService[:]}\OperatorTok{=}\NormalTok{AdoptionReturn[}\DecValTok{5}\NormalTok{]}
\NormalTok{        Adopted[:]}\OperatorTok{=}\NormalTok{AdoptionReturn[}\DecValTok{6}\NormalTok{]}
\NormalTok{        CumulativeProfit[:]}\OperatorTok{=}\NormalTok{AdoptionReturn[}\DecValTok{7}\NormalTok{]}
\NormalTok{        PanelFIT[:]}\OperatorTok{=}\NormalTok{AdoptionReturn[}\DecValTok{8}\NormalTok{]}
        \CommentTok{\#Calculate aggregate statistics}
\NormalTok{        StatisticsReturn}\OperatorTok{=}\NormalTok{CalculateStatistics(\_PV}\OperatorTok{=}\NormalTok{PV[:,period].copy(),\_Income}\OperatorTok{=}\NormalTok{Income[:,period].copy(),\_Debt}\OperatorTok{=}\NormalTok{Debt.copy(),\_Liquidity}\OperatorTok{=}\NormalTok{Liquidity[:,period].copy(),\_ExpectedPrice}\OperatorTok{=}\NormalTok{ExpectedPrice[:,}\DecValTok{0}\NormalTok{].copy(),\_Feasible}\OperatorTok{=}\NormalTok{Feasible.copy(),\_Rationed}\OperatorTok{=}\NormalTok{Rationed.copy())}
\NormalTok{        AverageIncome[period]}\OperatorTok{=}\NormalTok{StatisticsReturn[}\DecValTok{0}\NormalTok{]}
\NormalTok{        AverageLiquidity[period]}\OperatorTok{=}\NormalTok{StatisticsReturn[}\DecValTok{1}\NormalTok{]}
\NormalTok{        AverageExpectedPrice[period]}\OperatorTok{=}\NormalTok{StatisticsReturn[}\DecValTok{2}\NormalTok{]}
\NormalTok{        AverageDebt[period]}\OperatorTok{=}\NormalTok{StatisticsReturn[}\DecValTok{3}\NormalTok{]}
\NormalTok{        AdoptionRate1[period]}\OperatorTok{=}\NormalTok{StatisticsReturn[}\DecValTok{4}\NormalTok{]}
\NormalTok{        AdoptionRate2[period]}\OperatorTok{=}\NormalTok{StatisticsReturn[}\DecValTok{5}\NormalTok{]}
\NormalTok{        RationedShare[period]}\OperatorTok{=}\NormalTok{StatisticsReturn[}\DecValTok{6}\NormalTok{]}
        \CommentTok{\#Calculate change in adoption rates}
        \ControlFlowTok{if}\NormalTok{ period}\OperatorTok{\textgreater{}}\DecValTok{0}\NormalTok{:}
\NormalTok{            ChangeAdoptionRate1[period]}\OperatorTok{=}\NormalTok{AdoptionRate1[period]}\OperatorTok{{-}}\NormalTok{AdoptionRate1[(period}\OperatorTok{{-}}\DecValTok{1}\NormalTok{)]}
\NormalTok{            ChangeAdoptionRate2[period]}\OperatorTok{=}\NormalTok{AdoptionRate2[period]}\OperatorTok{{-}}\NormalTok{AdoptionRate2[(period}\OperatorTok{{-}}\DecValTok{1}\NormalTok{)]}
        \CommentTok{\#Calculate PV ownership rate by decile    }
\NormalTok{        PVDecileReturn}\OperatorTok{=}\NormalTok{PVDeciles(\_Income\_p}\OperatorTok{=}\NormalTok{Income\_p[:,period].copy(),\_PV}\OperatorTok{=}\NormalTok{PV[:,period].copy(),\_Positions\_d}\OperatorTok{=}\NormalTok{Positions\_d.copy(),\_PVOwnershipDeciles}\OperatorTok{=}\NormalTok{PVOwnershipDeciles.copy(),\_Deciles}\OperatorTok{=}\NormalTok{Deciles.copy(),\_DecileMembers}\OperatorTok{=}\NormalTok{DecileMembers.copy())}
\NormalTok{        PVOwnershipDeciles[:]}\OperatorTok{=}\NormalTok{PVDecileReturn[}\DecValTok{0}\NormalTok{]}
\NormalTok{        AdoptionRateDeciles[period]}\OperatorTok{=}\NormalTok{PVOwnershipDeciles}
        \CommentTok{\#Increment year}
\NormalTok{        year}\OperatorTok{=}\NormalTok{year}\OperatorTok{+}\DecValTok{1}

    \CommentTok{\#Save aggregate statistics and ownership rates by decile}
\NormalTok{    Stats}\OperatorTok{=}\NormalTok{np.column\_stack((AverageIncome,AverageLiquidity,AverageExpectedPrice,AverageDebt,AdoptionRate1,AdoptionRate2,ChangeAdoptionRate1,ChangeAdoptionRate2,RationedShare,Price[}\DecValTok{2}\NormalTok{:}\BuiltInTok{len}\NormalTok{(Price)]))}
\NormalTok{    Stats}\OperatorTok{=}\NormalTok{pd.DataFrame(data}\OperatorTok{=}\NormalTok{Stats,columns}\OperatorTok{=}\NormalTok{[}\StringTok{"AverageIncome"}\NormalTok{,}\StringTok{"AverageLiquidity"}\NormalTok{,}\StringTok{"AverageExpectedPrice"}\NormalTok{,}\StringTok{"AverageDebt"}\NormalTok{,}\StringTok{"AdoptionRate1"}\NormalTok{,}\StringTok{"AdoptionRate2"}\NormalTok{,}\StringTok{"ChangeAdoptionRate1"}\NormalTok{,}\StringTok{"ChangeAdoptionRate2"}\NormalTok{,}\StringTok{"RationedShare"}\NormalTok{,}\StringTok{"Price"}\NormalTok{])}
\NormalTok{    Stats\_m}\OperatorTok{=}\NormalTok{pd.DataFrame(data}\OperatorTok{=}\NormalTok{AdoptionRateDeciles,columns}\OperatorTok{=}\NormalTok{[}\StringTok{"1"}\NormalTok{,}\StringTok{"2"}\NormalTok{,}\StringTok{"3"}\NormalTok{,}\StringTok{"4"}\NormalTok{,}\StringTok{"5"}\NormalTok{,}\StringTok{"6"}\NormalTok{,}\StringTok{"7"}\NormalTok{,}\StringTok{"8"}\NormalTok{,}\StringTok{"9"}\NormalTok{,}\StringTok{"10"}\NormalTok{])}
\NormalTok{    filename}\OperatorTok{=}\StringTok{\textquotesingle{}outputSolar/out\_\textquotesingle{}}
\NormalTok{    filename}\OperatorTok{+=}\NormalTok{runname}
\NormalTok{    filename}\OperatorTok{+=}\StringTok{"\_"}
\NormalTok{    filename}\OperatorTok{+=}\BuiltInTok{str}\NormalTok{(seed)}
\NormalTok{    filename}\OperatorTok{+=}\StringTok{\textquotesingle{}.csv\textquotesingle{}}
\NormalTok{    filename2}\OperatorTok{=}\StringTok{\textquotesingle{}outputSolar/out\_deciles\_\textquotesingle{}}
\NormalTok{    filename2}\OperatorTok{+=}\NormalTok{runname}
\NormalTok{    filename2}\OperatorTok{+=}\StringTok{"\_"}
\NormalTok{    filename2}\OperatorTok{+=}\BuiltInTok{str}\NormalTok{(seed)}
\NormalTok{    filename2}\OperatorTok{+=}\StringTok{\textquotesingle{}.csv\textquotesingle{}}
\NormalTok{    Stats.to\_csv(filename)}
\NormalTok{    Stats\_m.to\_csv(filename2)}
\end{Highlighting}
\end{Shaded}

\hypertarget{section}{%
\section{}\label{section}}



\end{document}
